%%%%%%%%%%%%%%%%%%%%%%%%%%%%%%%%%%%%%%%%%%%%%%%%%%%%%%%%%%%%%%%%
\part{Computation}

The concept of computation that emerged in the mid-19th century and
crystallized in the 1930s is a fundamentally new concept.  It is as
basic as math and logic; it effectively extends the classic quadrivium
(geometry, arithmetic, astronomy, and music.)

\begin{remark}
  Why important for intro to stats?  Mainly for historical reasons;
  both effective procedure and axiom of choice emerged at roughly the
  same time.  Also, an understanding of effective proc sharpens
  understanding of choice and randomness.

  Practically, the emergence of statistics in the 20th century as the
  lingua-franca of empirical science was heavily dependent on
  computation technologies.  For example, today Bayesian statistics is
  all the rage; this only became possible with the advent of powerful
  but inexpensive computing devices.  So in practice learning
  statistics means learning computation techniques.  Or, learning to
  think statistically requires skill in thinking computationally; in
  practice this means learning how to use statistical software
  packages.  Unfortunately, one can learn to use such packages without
  understanding; in fact that is a permanent hazard of statistics
  education.  Part of the purpose here is to integrate statistical and
  computational thinking, so that statistical software (languages)
  become conceptually transparent.

  Example: R's idiosyncratic terminology, such as ``mode'' for
  ``type'', ``factor'' for ``category'', etc.  Understanding
  underlying computational concepts such as type allows one to ``see
  through'' some of the ad-hoc idiosyncrasies of particular
  statistical software languages to the underlying statistical
  concepts.
\end{remark}


%%%%%%%%%%%%%%%%%%%%%%%%
\chapter{Computability and Decideability}

\section{Effective Procedure}


%%%%%%%%%%%%%%%%%%%%%%%%%%%%%%%%%%%%%%%%%%%%%%%%%%%%%%%%%%%%%%%%
\chapter{Computation and Proof}

On the amazing convergence of proof theory and computability theory.

