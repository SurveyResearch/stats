\part{Modeling}

\begin{ednote}
  Modeling the world as data; it really should be called ``world
  modeling''.  E.g. in industry, a ``standard data model'' is a
  standard way of representing some domain of interest, such as
  electrical connectors or transmission lines.  TODO: clear, simple
  examples.  Maybe \href{http://www.pods.org/4/The PODS Data
    Model/}{Pipeline Open Data Standard} (PODS)?  ``The PODS Pipeline Data Model provides the database architecture pipeline operators use to store critical information and analysis data about their pipeline systems, and manage this data geospatially in a linear-referenced database which can then be visualized in any GIS platform.  The PODS Pipeline Data Model houses the asset information, inspection, integrity management, regulatory compliance, risk analysis, history, and operational data that pipeline companies have deemed mission-critical to the successful management of natural gas and hazardous liquids pipelines.''

  TODO: how is this concept of data model related to statistical
  notions of ``model'' and modeling?
\end{ednote}


