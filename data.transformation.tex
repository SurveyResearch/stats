\part{Transformation}

\begin{ednote}
  Why separate Data Transformation from Data Management?  Because you
  might need ad-hoc transformations.  You might need to clean data
  before archiving it, but such data munging is conceptually distinct
  from data management.
\end{ednote}

\href{http://www.unece.org/stats/editing.html}{UNECE Statistical Data Editing}

\href{http://www.unece.org/fileadmin/DAM/stats/publications/editingglossary.pdf}{UNECE GLOSSARY OF TERMS ON STATISTICAL DATA EDITING}

%%%%%%%%%%%%%%%%%%%%%%%%%%%%%%%%
\chapter{Cleaning}

Cleaning: making sure all the data is correctly typed, violates no semantic constraints, etc.

A critical part of data cleaning is what to do with bad data.  There
are basically two options: ignore/discard, or impute.

%%%%%%%%%%%%%%%%%%%%%%%%%%%%%%%%
\chapter{Normalization}

Normalization: transforming ``clean'' data into some desired form.
This includes things like renaming columns, converting NA codes into
the language-specific data (e.g. in R, converting 99 to NA), as well
as more statistickish things like mapping ranges of metric variables
to categories (e.g. ages 18-40 to ``Young'', 41 to 64 as
``Middle-aged'', and 65 and over to ``Old'').

NB: ``normalization'' is a bit of a misnomer here; it comes from
formal logic where it refers to the notion that there is a single
``normal'' form for each kind of data (e.g. '5' is the normal form of
'2+3', '1+4', etc.)  Here we use it to refer to some desired form, so
that to normalize data means to make sure it all has the desired form.
There is no implication that any one form can be privileged as the
normal form.  For example, so-called nominal data can be ``renamed''
at will, and no naming scheme is inherently better than any other -
'A, B, C' is neither better nor worse than 'Foo, Bar, Baz' or '1, 2,
3'.

\begin{ednote}
The discussion here uses a generic vocab, to be explained in the
intro.  For example, instead of ``factors'' and ``levels'' (R), we use
a more mathematical idiom and talk of enumerations and elements.  Etc.
(TODO: think about using type-theoretic language, so instead of
enumerations, ``product types'', etc.)
\end{ednote}

\section{Typical Patterns}

\begin{itemize}
\item rename a column
\item rename the elements of an enumeration
\item etc.
\end{itemize}

%%%%
\section{Alpha-conversion (aka Renaming)}
\label{sect:datarenaming}

%%%%%%%%
\subsection{Vars}
\label{sect:datavarrrenaming}

That is, renaming a variable rather than its values.  The latter is called recoding.

%%%%%%%%
\section{Recoding}
\label{sect:recodingr}

``Recoding'' means (here at least) replacing data values.
Technically, renaming is a form of recoding, but recoding can also
involve replacing data with different values.  (Which makes imputation
a form of recoding.)  E.g. we might want to increment the values in a
column by 1.

Technically, recoding essentially involves mapping over the col or row.

\href{http://www.uni-kiel.de/psychologie/rexrepos/posts/recode.html}{Recode variables} (examples)

\href{http://www.statmethods.net/management/variables.html}{New vars, recoding vars, renaming vars}

\href{http://www.cookbook-r.com/Manipulating\_data/Recoding\_data/}{Recoding vars: categorical, continuous, new}

\href{http://rprogramming.net/recode-data-in-r/}{Recode data in R: replacement, recoding}

\href{http://rwiki.sciviews.org/doku.php?id=tips:data-frames:recode_column}{Recode one column, output values into another column: transform(), replace()}

\href{http://rprogramming.net/recode-data-in-r/}{The Recode Command From the Package Car}

%%%%
\section{Data Derivation}
\label{sect:newvars}

Sometimes it is useful to derive new data from old; for example, if we
have a firstname and a lastname column, we might want to derive a
fullname column.

``copy of an existing field. Sometimes you don’t want to recode data but instead just want another column containing the same data.''

\href{http://rprogramming.net/recode-data-in-r/}{Recode into A New Field Using Data From An Existing field And Criteria from Another Field}

%%%%%%%%%%%%%%%%%%%%%%%%%%%%%%%%
\chapter{Imputation}

\begin{ednote}
Technically we might categorize imputation as a species of either
cleaning or normalization; but on the other hand it has its own
techniques, some of which are quite sophisticated, so we grant it the
dignity of a separate chapter.
\end{ednote}

%%%%%%%%%%%%%%%%%%%%%%%%%%%%%%%%%%%%%%%%%%%%%%%%%%%%%%%%%%%%%%%%
\chapter{Anonymization}

A major issue for Social Science data.

\href{http://ico.org.uk/for\_organisations/data\_protection/topic\_guides/anonymisation}{Anonymisation:
  managing data protection risk code of practice} (UK ICO)

\href{http://www.unece.org/stats/confidentiality.html}{UNECE Statistical confidentiality and disclosure protection}

\section{Disclosure Risk Analysis}

%%%%%%%%%%%%%%%%%%%%%%%%%%%%%%%%%%%%%%%%%%%%%%%%%%%%%%%%%%%%%%%%
\chapter{Filter, Map, Reduce}
\label{filtermapreduce}

%%%%
\section{Data Filtering}
\label{sect:datafiltering}

E.g. removing columns, rows, cells.

%%%%
\section{Data Mapping}
\label{sect:datamapping}

%%%%
\section{Data Reduction}
\label{sect:datareduction}


%%%%%%%%%%%%%%%%%%%%%%%%%%%%%%%%%%%%%%%%%%%%%%%%%%%%%%%%%%%%%%%%
\chapter{General Munging}

ref:  the \href{http://www.padsproj.org/index.html}{PADS project}

\href{http://r4stats.com/examples/data-management/}{Data Management} - a comparison of the commands used for common data ``management'' (actually, munging) tasks in R, SAS, SPSS and Stata.

See also \href{http://openrefine.org/}{OpenRefine} ``a powerful tool for working with messy data, cleaning it, transforming it from one format into another, extending it with web services, and linking it to databases''

%%%%%%%%%%%%%%%%%%%%%%%%%%%%%%%%
\section{Misc}

\begin{remark}
  examples taken from R, so far
\end{remark}

\href{http://plyr.had.co.nz/}{plyr}: ``plyr is a set of tools for a
common set of problems: you need to split up a big data structure into
homogeneous pieces, apply a function to each piece and then combine
all the results back together''


\href{http://rprogramming.net/r-data-manipulation/}{R Data Manipulation}
\href{http://www.cookbook-r.com/Manipulating\_data/}{Manipulating Data}


%%%%%%%%
\section{Replacing data}
\label{sect:datareplacement}

``replace the data in an existing field when you want to replace the data for every row (no criteria).'' \url{http://rprogramming.net/recode-data-in-r/}

%%%%%%%%%%%%%%%%%%%%%%%%%%%%%%%%
\section{Subsetting Data}
\label{sect:datasubsetting}

\href{http://rprogramming.net/subset-data-in-r/}{Subset data in R}

%%%%%%%%%%%%%%%%%%%%%%%%%%%%%%%%
\section{Weights}
\label{sect:dataweighting}

%%%%%%%%%%%%%%%%%%%%%%%%%%%%%%%%
\section{Aggregation}
\label{sect:dataaggregate}

\href{http://rprogramming.net/aggregate-data-in-r-using-data-table/}{Aggregate
  Data in R Using data.table} - here ``aggregate'' seems to be
synonymous with ``compute statistic of''.  Misnomer; an aggregate is not a summary.

