%%!TEX TS-program = xelatexmk
\documentclass{tufte-handout}
%% % For syntax trees
%% \usepackage{qtree}

\usepackage{fontspec,xltxtra,xunicode}
\usepackage{fontspec}
\defaultfontfeatures{Scale=MatchLowercase}

%% \setmainfont[Mapping=tex-text]{Times New Roman}
%% \setromanfont[Mapping=tex-text]{Times New Roman}
%% \setsansfont[Mapping=tex-text]{Arial}

\setmainfont[Mapping=tex-text]{TeX Gyre Pagella}
%% \setromanfont[Mapping=tex-text]{TeX Gyre Pagella}
%% \setsansfont[Mapping=tex-text]{TeX Gyre Heros}

%% \setmainfont[Mapping=tex-text]{Minion Pro}
%% \setromanfont[Mapping=tex-text]{Minion Pro}
%% \setsansfont[Mapping=tex-text]{TeX Gyre Heros}

%% \setmonofont{Courier}

%% Bugfix: see https://code.google.com/p/tufte-latex/issues/detail?id=64
% Set up the spacing using fontspec features
\renewcommand\allcapsspacing[1]{{\addfontfeature{LetterSpace=15}#1}}
\renewcommand\smallcapsspacing[1]{{\addfontfeature{LetterSpace=10}#1}}

%% \defaultfontfeatures{Mapping=tex-text}
%% \setromanfont[Mapping=tex-text]{TeX Gyre Schola}
%% \setsansfont[Scale=MatchLowercase,Mapping=tex-text]{TeX Gyre Heros}
%% \setmonofont[Scale=MatchLowercase]{TeX Gyre Cursor}

%% \TeXXeTstate=1
%% \newfontfamily{\sbl}[Script=Arabic]{Scheherazade}

%% \usepackage{amsmath}
%% \usepackage{booktabs} % for tables
%% % Style for linguistic terms
%% \newcommand{\jgn}[1]{\textbf{\textsc{#1}}}
%% %\newcommand{\jgn}[1]{\textsc{#1}}
%% % Linguistic environments
%% \usepackage{covington}
%% % Rename “Figure” to “Tree”
%% %% \renewcommand{\figurename}{Tree}
%% %\renewcommand{\tablename}{Tree}

\usepackage[toc,page,header]{appendix}
\usepackage{pdfpages}

\usepackage[usenames]{xcolor}

\usepackage{csquotes}
\usepackage{changepage}

\usepackage{float}
%% \floatstyle{boxed}
%% \restylefloat{figure}


\usepackage{graphicx}
%% \usepackage{caption}
%% \usepackage{subcaption}

%% \usepackage{tikz}
\usepackage{pgfplots}
\pgfplotsset{compat=1.9}

\usetikzlibrary{arrows,shapes,patterns,backgrounds,spy}


%% %%%<
\usepackage{verbatim}
%% \usepackage[active,tightpage]{preview}
%% \setlength\PreviewBorder{0pt}%
%% %%%>


\begin{comment}
:Title: PGFPlots examples

\end{comment}

\usepackage{xifthen}

%%%% macros %%%%

\lhead{}
\chead{}
\rhead{\scshape Distributions}
\lfoot{Stats}
\cfoot{Probs}
\rfoot{\thepage}
\renewcommand{\headrulewidth}{0.4pt}
\renewcommand{\footrulewidth}{0.4pt}

\makeatletter
\newcommand*{\centerfloat}{%
  \parindent \z@
  \leftskip \z@ \@plus 1fil \@minus \textwidth
  \rightskip\leftskip
  \parfillskip \z@skip}
\makeatother

\renewcommand{\allcapsspacing}[1]{{\addfontfeature{LetterSpace=20.0}#1}}
\renewcommand{\smallcapsspacing}[1]{{\addfontfeature{LetterSpace=5.0}#1}}
\renewcommand{\textsc}[1]{\smallcapsspacing{\textsmallcaps{#1}}}
\renewcommand{\smallcaps}[1]{\smallcapsspacing{\scshape\MakeTextLowercase{#1}}}

%% \renewcommand\allcapsspacing[1]{{\addfontfeature{LetterSpace = 15}#1}}
%% \renewcommand\smallcapsspacing[1]{{\addfontfeature{LetterSpace = 8}\scshape#1}}

%%%%%%%%%%%%%%%%
\title{Probability Distributions}
\author{G. A. Reynolds}
\date{\today}

%%%%%%%%%%%%%%%%%%%%%%%%%%%%%%%%%%%%%%%%%%%%%%%%%%%%%%%%%%%%%%%%
\begin{document}

\maketitle
\nocite{*}

\tableofcontents
%% \listoffigures

\section{Normal (Gaussian) Distributions}

Section about the normal dist

\pgfplotsset{width=10cm,compat=1.9}


\pgfmathdeclarefunction{gausse}{2}{%
  \pgfmathparse{1/(#2*sqrt(2*pi))*exp(-((x-#1)^2)/(2*#2^2))}%
}

\begin{figure}[h!]
\begin{tikzpicture}
\begin{axis}[every axis plot post/.append style={
  mark=none,domain=-2:3,samples=50,smooth}, % All plots: from -2:2, 50 samples, smooth, no marks
  axis x line*=bottom, % no box around the plot, only x and y axis
  axis y line*=left, % the * suppresses the arrow tips
  height=3cm,
  enlargelimits=upper] % extend the axes a bit to the right and top
  \addplot {gausse(0,0.5)};
  \addplot {gausse(1,0.75)};
\end{axis}
\end{tikzpicture}
\end{figure}

\section{B}

Look! another section!

\begin{figure}[h!]

\pgfmathdeclarefunction{kauss}{3}{%
  \pgfmathparse{1/(#3*sqrt(2*pi))*exp(-((#1-#2)^2)/(2*#3^2))}%
}

\begin{tikzpicture}
\begin{axis}[
  no markers,
  domain=0:6,
  samples=100,
  ymin=0,
  ymax=1,
  axis lines*=left,
  xlabel=$x$,
  every axis y label/.style={at=(current axis.above origin),anchor=south},
  every axis x label/.style={at=(current axis.right of origin),anchor=west},
  height=5cm,
  width=12cm,
  xtick=\empty,
  ytick=\empty,
  enlargelimits=false,
  clip=false,
  axis on top,
  grid = major
  %% hide y axis
  ]

 \addplot [very thick,cyan!50!black] {kauss(x, 3, 1)};

\pgfmathsetmacro\valueA{kauss(1,3,1)}
\pgfmathsetmacro\valueB{kauss(2,3,1)}
\pgfmathsetmacro\valueC{kauss(3,3,1)}
\draw [gray] (axis cs:1,0) -- (axis cs:1,1) %\valueA)
    (axis cs:5,0) -- (axis cs:5,1); %\valueA);
\draw [gray] (axis cs:2,0) -- (axis cs:2,1) %\valueB)
    (axis cs:4,0) -- (axis cs:4,1); % \valueB);
\draw [black] (axis cs:3,0) -- (axis cs:3,1) ;
\draw [yshift=2.6cm, latex-latex](axis cs:2, 0) -- node [fill=white] {$0.683$} (axis cs:4, 0);
\draw [yshift=2.0cm, latex-latex](axis cs:1, 0) -- node [fill=white] {$0.954$} (axis cs:5, 0);

\node[below] at (axis cs:1, 0)  {$\mu - 2\sigma$};
\node[below] at (axis cs:2, 0)  {$\mu - \sigma$};
\node[below] at (axis cs:3, 0)  {$\mu$};
\end{axis}

\end{tikzpicture}
\caption{another one}
\end{figure}

\section{C}

{\scshape Third time's a charm!}  Let's go.

\pgfmathdeclarefunction{kauss}{3}{%
  \pgfmathparse{1/(#3*sqrt(2*pi))*exp(-((#1-#2)^2)/(2*#3^2))}%
}

%% \pgfdeclarelayer{background layer}
%% \pgfdeclarelayer{foreground layer}
%% \pgfsetlayers{background layer,main,foreground layer}

\begin{figure}[h!]
\begin{tikzpicture} % [show background rectangle]

\tikzset{
every pin/.style={fill=yellow!50!white,rectangle,rounded corners=3pt,font=\tiny},
small dot/.style={fill=black,circle,scale=0.3}
}

\begin{axis}[
  %% legend style={
  %%   cells={anchor=east},
  %%   legend pos=outer north east,
  %% },
  %% title={$\mu=3$, $\sigma=1$}],
  clip=false,
  no markers,
  domain=0:6,
  samples=100,
  ymin=0,
  ymax=1,
  xmin=0,
  xmax=6,
  %% axis lines*=left,
  %% xlabel=$x$,
  %% every axis y label/.style={at=(current axis.above origin),anchor=south},
  %% every axis x label/.style={at=(current axis.below),anchor=south},
  xlabel={$x$},
  height=5cm,
  width=12cm,
  %% xtick=\empty,
  %% ytick=\empty,
  %% enlargelimits=false,
  %% clip=false,
  %% axis on top,
  %% grid = major
  %% hide y axis
  ]

 \addplot [very thick,cyan!50!black] {kauss(x, 3, 1)};

%% \node (25.4,.5) {$\mu=3$} ;
%% \node[small dot,pin=-30:{$(1,1)$}] at (axis description cs:1,1) {};

\pgfmathsetmacro\valueA{kauss(1,3,1)}
\pgfmathsetmacro\valueB{kauss(2,3,1)}
\pgfmathsetmacro\valueC{kauss(3,3,1)}

\draw [gray] (axis cs:1,0) -- (axis cs:1,1) %\valueA)
    (axis cs:5,0) -- (axis cs:5,1); %\valueA);
\draw [gray] (axis cs:2,0) -- (axis cs:2,1) %\valueB)
    (axis cs:4,0) -- (axis cs:4,1); % \valueB);
\draw [black] (axis cs:3,0) -- (axis cs:3,1) ;
\draw [yshift=2.6cm, latex-latex](axis cs:2, 0) -- node [fill=white] {$0.683$} (axis cs:4, 0);
\draw [yshift=2.0cm, latex-latex](axis cs:1, 0) -- node [fill=white] {$0.954$} (axis cs:5, 0);

\node[above] at (axis cs:1, 1)  {$\mu - 2\sigma$};
\node[above] at (axis cs:2, 1)  {$\mu - \sigma$};
\node[above] at (axis cs:3, 1)  {$\mu$};

\end{axis}

\end{tikzpicture}
\caption{A Gaussian (Normal) Curve

$\mu=3$

$\sigma=1$}
\end{figure}

And another figure

\begin{figure}[h!]
\begin{tikzpicture} % [show background rectangle]

\tikzset{
every pin/.style={fill=yellow!50!white,rectangle,rounded corners=3pt,font=\tiny},
small dot/.style={fill=black,circle,scale=0.3}
}

\begin{axis}[
  %% legend style={
  %%   cells={anchor=east},
  %%   legend pos=outer north east,
  %% },
  %% title={$\mu=3$, $\sigma=1$}],
  clip=false,
  no markers,
  domain=-3:3,
  samples=100,
  ymin=0,
  ymax=1,
  xmin=-3,
  xmax=3,
  %% axis lines*=left,
  %% xlabel=$x$,
  %% every axis y label/.style={at=(current axis.above origin),anchor=south},
  %% every axis x label/.style={at=(current axis.below),anchor=south},
  xlabel={$x$},
  height=5cm,
  width=12cm,
  %% xtick=\empty,
  %% ytick=\empty,
  %% enlargelimits=false,
  %% clip=false,
  %% axis on top,
  %% grid = major
  %% hide y axis
  ]

 \addplot [very thick,cyan!50!black] {kauss(x, 0, 1)};

%% \node (25.4,.5) {$\mu=3$} ;
%% \node[small dot,pin=-30:{$(1,1)$}] at (axis description cs:1,1) {};

\pgfmathsetmacro\valueA{kauss(1,0,1)}
\pgfmathsetmacro\valueB{kauss(2,0,1)}
\pgfmathsetmacro\valueC{kauss(3,0,1)}

\draw [gray] (axis cs:-1,0) -- (axis cs:-1,1)
    (axis cs:1,0) -- (axis cs:1,1);
\draw [gray] (axis cs:-2,0) -- (axis cs:-2,1)
    (axis cs:2,0) -- (axis cs:2,1);
\draw [black] (axis cs:0,0) -- (axis cs:0,1) ;
\draw [yshift=2.6cm, latex-latex](axis cs:-1, 0) -- node [fill=white] {$0.683$} (axis cs:1, 0);
\draw [yshift=2.0cm, latex-latex](axis cs:-2, 0) -- node [fill=white] {$0.954$} (axis cs:2, 0);

\node[above] at (axis cs:1, 1)  {$\mu - 2\sigma$};
\node[above] at (axis cs:2, 1)  {$\mu - \sigma$};
\node[above] at (axis cs:3, 1)  {$\mu$};

\end{axis}

\end{tikzpicture}
\caption{A Gaussian (Normal) Curve

$\mu=0$

$\sigma=1$}
\end{figure}

%%%%%%%%%%%%%%%%
\begin{figure}[h!]
\begin{tikzpicture} % [show background rectangle]

\tikzset{
every pin/.style={fill=yellow!50!white,rectangle,rounded corners=3pt,font=\tiny},
small dot/.style={fill=black,circle,scale=0.3}
}

\begin{axis}[
  %% legend style={
  %%   cells={anchor=east},
  %%   legend pos=outer north east,
  %% },
  %% title={$\mu=3$, $\sigma=1$}],
  clip=false,
  no markers,
  domain=-1:1,
  samples=100,
  ymin=0,
  ymax=1,
  xmin=-1,
  xmax=1,
  %% axis lines*=left,
  %% xlabel=$x$,
  %% every axis y label/.style={at=(current axis.above origin),anchor=south},
  %% every axis x label/.style={at=(current axis.below),anchor=south},
  xlabel={$x$},
  height=5cm,
  width=12cm,
  %% xtick=\empty,
  %% ytick=\empty,
  %% enlargelimits=false,
  %% clip=false,
  %% axis on top,
  %% grid = major
  %% hide y axis
  ]

 \addplot [very thick,cyan!50!black] {kauss(x, 0, .3)};

%% \node (25.4,.5) {$\mu=3$} ;
%% \node[small dot,pin=-30:{$(1,1)$}] at (axis description cs:1,1) {};

\pgfmathsetmacro\valueA{kauss(1,0,.3)}
\pgfmathsetmacro\valueB{kauss(2,0,.3)}
\pgfmathsetmacro\valueC{kauss(3,0,.3)}

\pgfmathsetmacro\onesig{.3}
\pgfmathsetmacro\twosig{.6}

\draw [gray] (axis cs:-\onesig,0) -- (axis cs:-\onesig,1)
    (axis cs:\onesig,0) -- (axis cs:\onesig,1);
\draw [gray] (axis cs:-\twosig,0) -- (axis cs:-\twosig,1)
    (axis cs:\twosig,0) -- (axis cs:\twosig,1);
\draw [black] (axis cs:0,0) -- (axis cs:0,1) ;

%% \draw [yshift=2.6cm, latex-latex](axis cs:-1, 0) -- node [fill=white] {$0.683$} (axis cs:1, 0);
%% \draw [yshift=2.0cm, latex-latex](axis cs:-2, 0) -- node [fill=white] {$0.954$} (axis cs:2, 0);

%% \node[above] at (axis cs:1, 1)  {$\mu - 2\sigma$};
%% \node[above] at (axis cs:2, 1)  {$\mu - \sigma$};
%% \node[above] at (axis cs:3, 1)  {$\mu$};

\end{axis}

\end{tikzpicture}
\caption{A Gaussian (Normal) Curve

$\mu=0$

$\sigma=.3$}
\end{figure}

%%%%%%%%%%%%%%%%
\begin{figure}[h!]
\begin{tikzpicture} % [show background rectangle]
\pgfmathsetmacro\onesig{.2}

\tikzset{
every pin/.style={fill=yellow!50!white,rectangle,rounded corners=3pt,font=\tiny},
small dot/.style={fill=black,circle,scale=0.3}
}

\begin{axis}[
  %% legend style={
  %%   cells={anchor=east},
  %%   legend pos=outer north east,
  %% },
  %% title={$\mu=3$, $\sigma=1$}],
  clip=false,
  no markers,
  domain=-.6:.6,
  samples=100,
  ymin=0,
  ymax=1,
  xmin=-.6,
  xmax=.6,
  %% axis lines*=left,
  %% xlabel=$x$,
  %% every axis y label/.style={at=(current axis.above origin),anchor=south},
  %% every axis x label/.style={at=(current axis.below),anchor=south},
  %% xlabel={$x$},
  height=3cm,
  %% width=8cm,
  %% xtick=\empty,
  %% ytick=\empty,
  %% enlargelimits=false,
  %% clip=false,
  %% axis on top,
  %% grid = major
  %% hide y axis
  ]

 \addplot [very thick,cyan!50!black] {kauss(x, 0, \onesig)};

%% \node (25.4,.5) {$\mu=3$} ;
%% \node[small dot,pin=-30:{$(1,1)$}] at (axis description cs:1,1) {};

\pgfmathsetmacro\valueA{kauss(\onesig,0,\onesig)}
\pgfmathsetmacro\valueB{kauss(2*\onesig,0,\onesig)}
\pgfmathsetmacro\valueC{kauss(0,0,\onesig)}

\draw [gray] (axis cs:-\onesig,0) -- (axis cs:-\onesig,\valueA)
    (axis cs:\onesig,0) -- (axis cs:\onesig,\valueA);
\draw [gray] (axis cs:2*-\onesig,0) -- (axis cs:2*-\onesig,\valueB)
    (axis cs:2*\onesig,0) -- (axis cs:2*\onesig,\valueB);
\draw [black] (axis cs:0,0) -- (axis cs:0,\valueC) ;

%% \draw [black,dashed] (axis cs:-1,\valueC) node[left] {\valueC} -- (axis cs:0,\valueC);

%% \draw [yshift=2.6cm, latex-latex](axis cs:-1, 0) -- node [fill=white] {$0.683$} (axis cs:1, 0);
%% \draw [yshift=2.0cm, latex-latex](axis cs:-2, 0) -- node [fill=white] {$0.954$} (axis cs:2, 0);

%% \node[above] at (axis cs:1, 1)  {$\mu - 2\sigma$};
%% \node[above] at (axis cs:2, 1)  {$\mu - \sigma$};
%% \node[above] at (axis cs:3, 1)  {$\mu$};

\end{axis}

\end{tikzpicture}
\caption{A Gaussian (Normal) Curve

$\mu=0$

$\sigma=.2$}
\end{figure}

\section{Espionage}

{\scshape The SPY environment is pretty cool.}  Just watch!

%%%%%%%%%%%%%%%%
\begin{figure}[h!]
\begin{tikzpicture} % [show background rectangle]
 [spy using outlines={circle, magnification=3, size=1cm, connect spies}]
\pgfmathsetmacro\onesig{.2}

\tikzset{
every pin/.style={fill=yellow!50!white,rectangle,rounded corners=3pt,font=\tiny},
small dot/.style={fill=black,circle,scale=0.3}
}

\begin{axis}[
  %% legend style={
  %%   cells={anchor=east},
  %%   legend pos=outer north east,
  %% },
  %% title={$\mu=3$, $\sigma=1$}],
  clip=false,
  no markers,
  domain=-.6:.6,
  samples=100,
  ymin=0,
  ymax=1,
  xmin=-.6,
  xmax=.6,
  %% axis lines*=left,
  %% xlabel=$x$,
  %% every axis y label/.style={at=(current axis.above origin),anchor=south},
  %% every axis x label/.style={at=(current axis.below),anchor=south},
  %% xlabel={$x$},
  height=3cm,
  %% width=8cm,
  %% xtick=\empty,
  %% ytick=\empty,
  %% enlargelimits=false,
  %% clip=false,
  %% axis on top,
  %% grid = major
  %% hide y axis
  ]

 \addplot [very thick,cyan!50!black] {kauss(x, 0, \onesig)};

%% \node (25.4,.5) {$\mu=3$} ;
%% \node[small dot,pin=-30:{$(1,1)$}] at (axis description cs:1,1) {};

\pgfmathsetmacro\valueA{kauss(\onesig,0,\onesig)}
\pgfmathsetmacro\valueB{kauss(2*\onesig,0,\onesig)}
\pgfmathsetmacro\valueC{kauss(0,0,\onesig)}

\draw [gray] (axis cs:-\onesig,0) -- (axis cs:-\onesig,\valueA)
    (axis cs:\onesig,0) -- (axis cs:\onesig,\valueA);
\draw [gray] (axis cs:2*-\onesig,0) -- (axis cs:2*-\onesig,\valueB)
    (axis cs:2*\onesig,0) -- (axis cs:2*\onesig,\valueB);
\draw [black] (axis cs:0,0) -- (axis cs:0,\valueC) ;

%% \draw [black,dashed] (axis cs:-1,\valueC) node[left] {\valueC} -- (axis cs:0,\valueC);

%% \draw [yshift=2.6cm, latex-latex](axis cs:-1, 0) -- node [fill=white] {$0.683$} (axis cs:1, 0);
%% \draw [yshift=2.0cm, latex-latex](axis cs:-2, 0) -- node [fill=white] {$0.954$} (axis cs:2, 0);

%% \node[above] at (axis cs:1, 1)  {$\mu - 2\sigma$};
%% \node[above] at (axis cs:2, 1)  {$\mu - \sigma$};
%% \node[above] at (axis cs:3, 1)  {$\mu$};

%% \coordinate (spypoint) at (axis cs:2*\onesig,\valueB);

%% \spy [red] on (5.5,0.1) in node at (3,1);

\coordinate (spypoint) at (axis cs:0.6,0);
\coordinate (magnifyglass) at (axis cs:0.5,2);

\end{axis}

\spy [blue, size=2.5cm] on (spypoint) in node[fill=blue!20] at (magnifyglass);

\end{tikzpicture}
\caption{A Gaussian (Normal) Curve

$\mu=0$

$\sigma=.2$}
\end{figure}

\end{document}
