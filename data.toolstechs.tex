\part{Tools \& Techniques}

\href{http://kieranhealy.org/files/misc/workflow-apps.pdf}{Choosing your workflow applications}

%%%%%%%%%%%%%%%%%%%%%%%%%%%%%%%%%%%%%%%%%%%%%%%%%%%%%%%%%%%%%%%%
\chapter{Overview}
%%%%%%%%%%%%%%%%%%%%%%%%%%%%%%%%

%%%%%%%%%%%%%%%%%%%%%%%%%%%%%%%%
\section{Standards}
\label{sect:standards}

\begin{description}
\item [OAIS]
\item [DDI]
\item [SMDX]
\end{description}

\href{http://www.ihsn.org/HOME/node/135}{IHSN Metadata Standards and Models}

%%%%%%%%%%%%%%%%%%%%%%%%%%%%%%%%
\section{Editors}
\label{sect:editors}

\begin{description}
\item [Emacs]
\item [Vi]
\item [Eclipse]
\item [etc]
\end{description}

%%%%%%%%%%%%%%%%%%%%%%%%%%%%%%%%
\section{Languages}
\label{sect:langs}

\begin{itemize}
\item \href{http://ieeexplore.ieee.org/xpl/articleDetails.jsp?arnumber=6228573}{Computing Trends Lead to New Programming Languages}
\item \href{http://cloudcomputing.sys-con.com/node/2265359}{Twelve New Programming Languages: Is Cloud Responsible?}
\item \href{http://r4stats.com/articles/popularity/}{The Popularity of Data Analysis Software} (2014) A very detailed data-driven analysis of the relative popularity of various languages.
\end{itemize}

\begin{remark}
  Organize by category?
\end{remark}

%%%%%%%%%%%%%%%%%%%%%%%%%%%%%%%%%%%%%%%%%%%%%%%%%%%%%%%%%%%%%%%%
\chapter{Standards}

\begin{ednote}
  TODO: standards v. data models
\end{ednote}

%%%%%%%%%%%%%%%%%%%%%%%%%%%%%%%%
\section{OAIS}
\label{sect:oais}

\href{http://public.ccsds.org/publications/archive/650x0m2.pdf}{Reference
  Model for an Open Archival Information System} (OAIS)

%%%%%%%%%%%%%%%%%%%%%%%%%%%%%%%%
\section{Statistical Standards}
\label{sect:statsstd}

\begin{itemize}
\item \href{http://unstats.un.org/unsd/iiss/List-of-Statistical-Standards.ashx}{UN
  Global Inventory of Statistical Standards}

\item \href{https://nces.ed.gov/statprog/standards.asp}{Statistical Standards Program} - National Center for Education Statistics

\item \href{https://stats.oecd.org/glossary/index.htm}{OECD Glossary of Statistical Terms}

\item \href{http://shop.bsigroup.com/Browse-By-Subject/Quality--Sampling/Full-list-of-statistical-standards/Statistical-interpretation-of-data/}{Standards for Statistical Interpretation of Data} (British Standards Institute)

\item \href{http://www.whitehouse.gov/omb/inforeg_statpolicy}{OMB Statistical Programs and Standards}

\end{itemize}


%%%%%%%%%%%%%%%%%%%%%%%%%%%%%%%%
\subsection{DDI}
\label{subsect:ddi}

%%%%%%%%%%%%%%%%%%%%%%%%%%%%%%%%
\subsection{SDMX}
\label{subsect:sdmx}

\href{http://sdmx.org/}{SDMX} - Statistical Data and Metadata eXchange

%%%%%%%%%%%%%%%%%%%%%%%%%%%%%%%%%%%%%%%%%%%%%%%%%%%%%%%%%%%%%%%%
\chapter{Other}

%%%%%%%%%%%%%%%%%%%%%%%%%%%%%%%%
\section{OCaml}
\label{sect:ocaml}

\href{http://www.ffconsultancy.com/products/ocaml_for_scientists/chapter1.html}{Objective Caml for Scientists}

%%%%%%%%%%%%%%%%%%%%%%%%%%%%%%%%
\section{F\#}
\label{sect:fsharp}

\href{http://fsharp.org/data-science/}{Using F\# for Data Science}

%%%%%%%%
\subsection{Experimental/Research}
\label{subs:expersch}

\begin{itemize}
\item \href{http://chapel.cray.com/overview.html}{Chapel}
\item \href{http://x10-lang.org/}{X10} IBM Research; ``Both its modern, type-safe sequential core and simple programming model for concurrency and distribution contribute to making X10 a high-productivity language in the HPC and Big Data spaces.''
\end{itemize}

%%%%%%%%%%%%%%%%%%%%%%%%%%%%%%%%%%%%%%%%%%%%%%%%%%%%%%%%%%%%%%%%
\chapter{J}

J is a hideous language.  But ``J is particularly strong in the
mathematical, statistical, and logical analysis of
data.''(\url{http://www.jsoftware.com/}) If you can stomach the
syntax, not to mention the metalanguage (an ``adverb'' is unary op? a
``monad'' is a unary function? really?), not to mention the mental
model, you may find it useful.

In J, all data is an array.  etc.

Here's an example.  The following produces all the factorizations of a
number:

\begin{verbatim}
ext=: [: ~. ,&.> , ;@:(tu&.>)
tu =: ] <@:(/:~)@:*"1 [ ^ </\"1@=@]
af =: ext/ @ q:
\end{verbatim}

Holy \href{http://www.muppetlabs.com/~breadbox/bf/}{Brainfuck}!!!  For comparison, here is Hello World
written in that infamous language:

\begin{verbatim}
  ++++++++[>++++[>++>+++>+++>+<<<<-]>+>+>->>+[<]<-]>>.>---.+++++++..+++.>>.<-.<.+++.------.--------.>>+.>++.
\end{verbatim}

The obvious question: is the benefit of becoming fluent in such a
syntax worth the cost?  More to the point: does it really provide
anything that you cannot find in some other less alienating language?

%%%%%%%%%%%%%%%%%%%%%%%%%%%%%%%%%%%%%%%%%%%%%%%%%%%%%%%%%%%%%%%%
\chapter{Commercial}

