\documentclass[reqno,12pt]{tufte-book}
%% \documentclass[reqno,12pt]{tufte-handout}

\usepackage[toc,page,header]{appendix}
\usepackage{csquotes}
\usepackage{amsmath}
\usepackage{setspace}
\numberwithin{equation}{subsection}
\usepackage{amsfonts}

%% these two go together!
\usepackage{framed}
\usepackage[standard,framed]{ntheorem}

\usepackage{zed-csp}


%\usepackage{geometry}                % See geometry.pdf to learn the layout options. There are lots.
%\geometry{letterpaper}                   % ... or a4paper or a5paper or ... 

\usepackage{epigraph}
\setlength{\epigraphwidth}{.8\textwidth}

\usepackage{xfrac}

%% \usepackage[
%% bibstyle=numeric,
%% citestyle=authoryear,
%% hyperref,
%% bibencoding=utf8,
%% backref=true,
%% backend=biber]{biblatex}

\usepackage{hyperref}
\hypersetup{
    bookmarks=true,         % show bookmarks bar?
    unicode=true,          % non-Latin characters in Acrobat’s bookmarks
    pdftoolbar=true,        % show Acrobat’s toolbar?
    pdfmenubar=true,        % show Acrobat’s menu?
    pdffitwindow=false,     % window fit to page when opened
    pdfstartview={FitH},    % fits the width of the page to the window
    pdftitle={Think Stats},
    pdfauthor={G. A. Reynolds},     % author
    pdfsubject={Statistics},   % subject of the document
    pdfcreator={G. A. Reynolds},   % creator of the document
    pdfproducer={G. A. Reynolds}, % producer of the document
    pdfkeywords={Statistics}, % list of keywords
    pdfnewwindow=true,      % links in new window
    colorlinks=true,       % false: boxed links; true: colored links
    linkcolor=blue,          % color of internal links
    citecolor=blue,        % color of links to bibliography
    filecolor=magenta,      % color of file links
    urlcolor=cyan           % color of external links
}

\usepackage{tikz}
%\usetikzlibrary{trees,snakes}

\title{Think Stats \\
\vspace{12pt}
\Large A Conceptual Introduction to Statistics \\
\vspace{12pt}
\large for the Skeptical, the Pessimistic, and the Mildly Disturbed}
\author{G. A. Reynolds}
%\date{}                                           % Activate to display a given date or no date

%%%%%%%%%%%%%%%%
%% customizations

\makeatletter
\let\runauthor\@author
\let\runtitle\@title
\makeatother

%% running headers
\newcommand{\changefont}{%
  \fontsize{7}{9.5}\selectfont
}
\fancypagestyle{plain}{
  \fancyhead[LO,LE]{\leftmark }
  \fancyhead[RO,RE]{\rightmark}
  \fancyfoot[CO,CE]{\thepage}
  \fancyfoot[LE]{\textsc{Think Statistics}}
  \fancyfoot[RO]{\textsc{Think Statistics}}
  \renewcommand{\headrulewidth}{0pt}
  \renewcommand{\footrulewidth}{0pt}
}
\pagestyle{plain}

\includeonly{%
mathlog
%% ,computation
%% ,probability
,statistics
}

%% macros
%% \newtheorem{principle}{Principle}
\newtheorem{axiom}{Axiom}
%% \newtheorem{cor}{Corollary}
%% \newtheorem{lem}{Lemma}
\newframedtheorem{ednote}{Ed. note}[section]

\newcommand\R{\ensuremath{\real}}

%%%%%%%%%%%%%%%%%%%%%%%%%%%%%%%%%%%%%%%%%%%%%%%%%%%%%%%%%%%%%%%%
\begin{document}
\maketitle
%% \nocite{*}

\begin{remark}
  Two tasks: conceptual and computational.  Understand stats first,
  then learn how to \emph{do} stats using software (statistics
  machines).
\end{remark}

\tableofcontents

%%%%%%%%%%%%%%%%%%%%%%%%
\chapter{Introduction}

What is statistics\footnote{test note}?

\epigraph{Modern statistics provides a quantitative technology for
  empirical science; it is a logic and methodology for the measurement
  of uncertainty and for an examination of the consequences of that
  uncertainty in the planning and interpretation of experimentation
  and observation.}{\citet{stigler_history_1986}}

\epigraph{If all sciences require measurement--and statistics is the
  logic of measurement--it follows that the history of statistics can
  encompass the history of all of
  science.}{\citet{stigler_history_1986}}


%%%%%%%%%%%%%%%%%%%%%%%%%%%%%%%%%%%%%%%%%%%%%%%%%%%%%%%%%%%%%%%%
%%%%%%%%%%%%%%%%%%%%%%%%%%%%%%%%%%%%%%%%%%%%%%%%%%%%%%%%%%%%%%%%
\part{Mathematics \& Logic}

%%%%%%%%%%%%%%%%%%%%%%%%
\chapter{Sets `n Stuff}

We use the Z Specification notation \cite{z-iso-13568}.

\begin{description}
\item [Sets] Membership, subset; family of sets
\item [Relations]
\item [Functions]
\item [Sequences]
\item [Multisets]
\end{description}

%%%%
\section{Sets}

Notation: extension v. comprehension


%%%%
\section{Relations}

%%%%
\section{Functions}


\subsection{Definition}

\begin{description}
\item [Function] Informally, a function is a set of ordered pairs.
  The Z specification says ``A function is a particular form of
  relation, where each domain element has only one corresponding range
  element.''\cite{z-iso-13568}%p [9]
\item [Function Extension] Since a function is a kind of set, it can
  be defined by explicitly listing its extension--all of its elements.
  For example, the function $f$ that maps each integer between $1$ and
  $3$ to itself can be expressed by writing out the complete list of
  its elements: ${f=\{(1,1),(2,2),(3,3)\}}$.
\item [Function Comprehension] A second way of defining a function is
  to express the ``rule'' that determines the elements it contains,
  without listing them explicitly.  For example ``the function that
  maps every number to itself'' defines the identity function.  Z provides
  two ways of doing this, one using standard set comprehension
  notation, the other using ``function construction'' notation.  See
  below.
\end{description}

\subsection{Notation}

The Z notation supports several ways of representing a function.  A
function extension expression may use ordered pair notation or maplet
notation.  The following example illustrates two ways to define the
function that maps each integer between $1$ and $3$, inclusive, to its
double.

\begin{alignat}{2}
  f &= \{(1,2),(2,4),(3,6)\} \\
  &= \{1\mapsto 2, 2\mapsto 4, 3\mapsto 6\}
\end{alignat}

\noindent More generally:

\begin{alignat}{2}
  \langle e_1,\ldots e_n\rangle &= \{(1,e_1),\ldots (3,e_n)\} \\
  &= \{1\mapsto e_1,\ldots 3\mapsto e_n\}
\end{alignat}

Z also supports two ways to define a function intensionally -- in
terms of a property rather than an explicit list of elements.

\begin{remark}
FIXME: set comprehension expression for functions:
\begin{alignat}{2}
  f &= \{x,y | y=2x @ (x,y)\} \\
  &= \{x,y | y=2x @ x\mapsto y\}
\end{alignat}
\end{remark}

\textit{Function construction} notation...
\begin{alignat}{2}
  f &= \{\lambda
\end{alignat}


%%%%
\section{Sequences}
\label{subs:sequences}

\begin{remark}
  Lay down the basic concepts, terminology, and notation, for later
  use in discussing sampling, etc.

  cf. Series
\end{remark}

\subsection{Definition}

\begin{description}
\item [Sequence] Informally, a sequence is an ordered set.  The Z
  specification says ``A sequence is a particular form of function,
  where the domain elements are all the natural numbers from 1 to the
  length of the sequence.''
\end{description}

\subsection{Notation}

The Z notation uses angle brackets to form a sequence expression:

\begin{alignat}{2}
  \langle e_1,\ldots e_n\rangle &= \{(1,e_1),\ldots (3,e_n)\} \\
  &= \{1\mapsto e_1,\ldots 3\mapsto e_n\}
\end{alignat}

%%%%%%%%%%%%%%%%%%%%%%%%
\section{Multisets}

We follow \citet{singh_overview_2007}, with some modifications.

\subsection{Definitions}

\begin{description}
\item [Multiset]
\item [Element]
\item [Carrier]  The carrier of an mset is the set from which its elements is drawn.
\item [Generator]  The generators of an mset are the elements of its carrier set.
\item [Multiplcity]  The multiplicity of an element of an mset is the number of times it ``occurs'' (``appears'', etc.).
\item [Cardinality] The cardinality (size) of an mset is the sum of
  the multiplicities of its elements.  The cardinality of the carrier of an mset is the number of elements it contains.
\end{description}

\subsection{Notation}

The mset containing one $a$, two $b$, and three $c$

\begin{equation}
  M = \{(a,1), (b,2), (c, 3)\}
\end{equation}

\noindent can be written as follows:

\begin{description}
\item [Multiplicative notation]  The following are equivalent:

\begin{itemize}
\item $[[a,b,b,c,c,c]]$
\item $[a,b,b,c,c,c]$
\item $[a,b,c]_{1,2,3}$
\item $[a^1,b^2,c^3]$
\item $[a1,b2,c3]$
\end{itemize}

Note order is irrelevant.

\item [Linear notation]  The following are equivalent:

\begin{itemize}
\item $[a]+2[b]+3[c]$
\item $\{[a],2[b],3[c]\}$ Note that this style combines multiplicative
  notation for mset elements with standard set notation ($\{\ldots\}$)
  for the mset itself.
\end{itemize}
\end{description}

One advantage of the linear notation is that it allows us to have
non-integral and non-positive multiplicities; for example, in
$\{[a],-0.5[b],\pi[c]\}$ element $a$ occurs once, $b$ occurs $-0.5$
times, and $c$ occurs $\pi$ times.

In addition, linear notation allows a concise variant somewhat like a
stem-and-leaf display:

$$\{[a],2[b],2[c],3[d]\} = \{1[a], 2[b,c], 3[c]\}$$

%%%%%%%%
\subsection{Stem and Leaf}

Multisets can be represented using stem-and-leaf tables:

%%%%%%%%%%%%%%%%%%%%%%%%%%%%%%%%
\section{Multisequences}

A multiset is a set of ordered pairs, and is therefore unordered.
Just as we can extend the concept of set to form a sequence, we can
extend the notion of multiset to form a multisequence.

\begin{description}
\item [Multisequence]  A multisequence is a sequence of multiset elements.
\end{description}

We use the same angle bracket notation we use for set sequences:

\begin{alignat}{2}
  \langle [a],2[b],3[c]\rangle &= \{(1, (a,1)), (2,(b,2)), (3,(3,c))\} \\
  &= \{1\mapsto (a,1), 2\mapsto (b,2), 3\mapsto (3,c)\} \\
  &= \langle a,b,b,c,c,c\rangle
\end{alignat}

Since multisets are unordered, we have

\begin{equation}
  \{[a],2[b],3[c]\} = \{2[b],[a],3[c]\} = \{3[c],2[b],[a]\} = \ldots
\end{equation}

Multisequences are ordered, so for example

\begin{equation}
  \langle [a],2[b],3[c]\rangle \neq \langle2[b],[a],3[c]\rangle
\end{equation}

\noindent since

\begin{equation}
  \{(1,(a,2), (2,(b,2), (3,(c,3)\}\neq \{(1,(b,2)), (2,(a,2)), (3,(c,3))\}
\end{equation}

\noindent alternatively

\begin{equation}
  \{(1\mapsto (a,2), 2\mapsto (b,2), 3\mapsto (c,3)\} 
  \neq \{(1\mapsto (b,2)), 2\mapsto (a,2)), 3\mapsto (c,3))\}
\end{equation}

Also:  $\langle a,b,b,c,c,c\rangle \neq [a,b,b,c,c,c]$

%%%%%%%%%%%%%%%%%%%%%%%%
\chapter{Counting and Combining}

\cite{berge_principles_1971}

What is combinatorics?

\blockquote{Combinatorics can rightly be called the mathematics of counting. More specifically, it
is the mathematics of the enumeration, existence, construction, and optimization questions
concerning finite sets. (Mazur, guided tour)}

\blockquote{The concept of configuration can be made mathematically precise by defining it as a mapping of a set of objects into a finite abstract set with a given structure; for example, a permutation of n objects is a “bijection of the set of n objects into the ordered set 1, 2, ..., n.” Nevertheless, one is only interested in mappings satisfying certain constraints.

Just as arithmetic deals with integers (with the standard operations), algebra deals with operations in general, analysis deals with functions, geometry deals with rigid shapes, and topology deals with continuity, so does combinatorics deal with configurations. Combinatorics counts, enu- merates,* examines, and investigates the existence of configurations with certain specified properties.
}

\section{Counting Principles}

\begin{remark}
  Most of these are just restatements of basic arithmetic, expressed
  in terms of doing things.  Why bother?  I suspect that thinking in
  terms of sequences of actions makes it easier to do combinatorics.
  Also, these ideas only seem to be articulated as principles in
  elementary texts for e.g. high school algebra.
\end{remark}

\begin{remark}
  But isn't combinatorics just the science of counting?
\end{remark}


\begin{description}
\item [Fundamental Principle of Counting]

\item [Principle of addition]: if there are $a$ ways of doing one thing
  and $b$ ways of doing another, and we cannot do both, then there are
  $a+b$ ways to choose one thing to do.  This is just a restatement of
  a set-theoretic definition of addition in terms of union of sets.

  In combinatoric texts something like this is more typical:

  \textit{Addition Rule}. If $A$ and $B$ are finite, disjoint sets,
  then $A \cup B$ is finite and $|A \cup B| = |A| + |B|$.

\item [Principle of multiplication]: if there are $a$ ways of doing
  one thing and $b$ ways of doing another, then there are $a\cdot b$
  ways of doing both.  This is a restatement of a set-theoretic
  definition of addition in terms of cartesian products.
\end{description}

%%%%%%%%%%%%%%%%%%%%%%%%%%%%%%%%%%%%%%%%%%%%%%%%%%%%%%%%%%%%%%%%
\chapter{Progressions and Means}

%%%%%%%%%%%%%%%%%%%%%%%%%%%%%%%%
\section{Arithmetic}
\label{sect:meanarith}

\begin{definition}[Arithmetic Progression]
  
\end{definition}

\begin{definition}[Arithmetic Mean]
  \[\dfrac{x_1+\cdots x_n}{n} = \dfrac{\sum_1^n x_n}{n} = \dfrac{\displaystyle\sum x}{n}\]
\end{definition}

%%%%%%%%%%%%%%%%%%%%%%%%%%%%%%%%
\section{Geometric}
\label{sect:meangeom}

\begin{definition}[Geometric Progression]
  
\end{definition}

\begin{definition}[Geometric Mean]
  \[\sqrt[n]{x_1\times\cdots\times x_n} = \sqrt[n]{\displaystyle\Pi_1^n x} = \sqrt[n]{\displaystyle\Pi x}\]
\end{definition}
%%%%%%%%%%%%%%%%%%%%%%%%%%%%%%%%
\section{Harmonic}
\label{sect:meanharmonic}

%%%%%%%%%%%%%%%%%%%%%%%%
\chapter{Choice and Chance}

\section{The Axiom of Choice}

The Axiom of Choice is of enormous important in mathematics generally.
Statistics is no exception; the significance of the axiom will become
especially apparent when we discuss the concept of random sample.

\blockquote{The principle of set theory known as the Axiom of Choice has been hailed as “probably the most interesting and, in spite of its late appearance, the most discussed axiom of mathematics, second only to Euclid's axiom of parallels which was introduced more than two thousand years ago” (Fraenkel, Bar-Hillel \& Levy 1973, §II.4).  The fulsomeness (\textit{sic}) of this description might lead those unfamiliar with the axiom to expect it to be as startling as, say, the Principle of the Constancy of the Velocity of Light or the Heisenberg Uncertainty Principle. But in fact the Axiom of Choice as it is usually stated appears humdrum, even self-evident. For it amounts to nothing more than the claim that, given any collection of mutually disjoint nonempty sets, it is possible to assemble a new set — a transversal or choice set — containing exactly one element from each member of the given collection. Nevertheless, this seemingly innocuous principle has far-reaching mathematical consequences — many indispensable, some startling — and has come to figure prominently in discussions on the foundations of mathematics. It (or its equivalents) have been employed in countless mathematical papers, and a number of monographs have been exclusively devoted to it.\cite{bell_axiom_2013}}

Often stated in terms of choice functions.

Variants:

AC1: 
Any collection of nonempty sets has a choice function.''

AC2: 
Any indexed collection of sets has a choice function.

Or relations:

\begin{axiom}[Axiom of Choice]
\label{ax:choice}
For every family $\mathcal{F}$ of nonempty disjoint sets there exists
a \textit{selector}, that is, a set $S$ that intersects every $F\in
\mathcal{F}$ in precisely one point.\cite{ciesielski_set_1997}
\end{axiom}

Transversal:  In a 1908 paper Zermelo introduced a modified form of AC. Let us call a transversal (or choice set) for a family of sets H any subset T ⊆ ∪H for which each intersection T ∩ X for X ∈ H has exactly one element.  As a very simple example, let H = {{0}, {1}, {2, 3}}. Then H has the two transversals {0, 1, 2} and {0, 1, 3}. A more substantial example is afforded by letting H be the collection of all lines in the Euclidean plane parallel to the x-axis. Then the set T of points on the y-axis is a transversal for H.

So we have choice functions and choice sets.


``Let us call Zermelo's 1908 formulation the combinatorial axiom of choice:

CAC: 
Any collection of mutually disjoint nonempty sets has a transversal.'' (bell)


The problem:


\blockquote{It is to be noted that AC1 and CAC for finite collections
  of sets are both provable (by induction) in the usual set
  theories. But in the case of an infinite collection, even when each
  of its members is finite, the question of the existence of a choice
  function or a transversal is problematic[4]. For example, as already
  mentioned, it is easy to come up with a choice function for the
  collection of pairs of real numbers (simply choose the smaller
  element of each pair). But it is by no means obvious how to produce
  a choice function for the collection of pairs of arbitrary sets of
  real numbers.\cite{bell_axiom_2013}}

Footnote: 
\blockquote{The difficulty here is amusingly illustrated by an anecdote due to Bertrand Russell. A millionaire possesses an infinite number of pairs of shoes, and an infinite number of pairs of socks. One day, in a fit of eccentricity, the millionaire summons his valet and asks him to select one shoe from each pair. When the valet, accustomed to receiving precise instructions, asks for details as to how to perform the selection, the millionaire suggests that the left shoe be chosen from each pair. Next day the millionaire proposes to the valet that he select one sock from each pair. When asked as to how this operation is to be carried out, the millionaire is at a loss for a reply, since, unlike shoes, there is no intrinsic way of distinguishing one sock of a pair from the other. In other words, the selection of the socks must be truly arbitrary.}

The axiom of choice and probability (randomness) are different
concepts.  See
http://math.stackexchange.com/questions/29381/picking-from-an-uncountable-set-axiom-of-choice

\section{Chance and Randomness}

See \cite{eagle_chance_2014}

Indeterminacy, disorder, chaos, stochastic process, etc.

chance of a process, randomness of its product

Chance: physical; randomness: mathematical?

``It is safest, therefore, to conclude that chance and randomness, while they overlap in many cases, are separate concepts.''\cite{eagle_chance_2014}

Process v. Product concepts

``Of course the terminology in common usage is somewhat slippery; it's not clear, for example, whether to count random sampling as a product notion, because of the connection with randomness, or as a process notion, because sampling is a process.''

\blockquote{The upshot of this discussion is that chance is a process notion, rather than being entirely determined by features of the outcome to which the surface grammar of chance ascriptions assigns the chance. For if there can be a single-case chance of ½ for a coin to land heads on a toss even if there is only one actual toss, and it lands tails, then surely the chance cannot be fixed by properties of the outcome ‘lands heads’, as that outcome does not exist.[2] The chance must rather grounded in features of the process that can produce the outcome: the coin-tossing trial, including the mass distribution of the coin and the details of how it is tossed, in this case, plus the background conditions and laws that govern the trial. Whether or not an event happens by chance is a feature of the process that produced it, not the event itself.\cite{eagle_chance_2014}}

\blockquote{a process conception of randomness makes nonsense of some obvious uses of ‘random’ to characterise an entire collection of outcomes of a given repeated process. This is the sense in which a random sample is random: it is an unbiased representation of the population from which it is drawn—and that is a property of the entire sample, not each individual member. While many random samples will be drawn using a random process, they need not be....To be sure that our sample is random, we may wish to use random numbers to decide whether to include a given individual in the sample; to that end, large tables of random digits have been produced, displaying no order or pattern (RAND Corporation 1955). This other conception of randomness, as attaching primarily to collections of outcomes, has been termed product randomness.(eagle)}

\blockquote{If the actual process that generate the sequences are perfectly deterministic, it may be that a typical product of that process is not random. But we are rather concerned to characterise which of all the possible sequences produced by any process whatsoever are random, and it seems clear that most of the ways an infinite sequence might be produced, and hence most of the sequences so produced, will be random.(eagle)}

...concentrate on the sequence of outcomes as independently given mathematical entities, rather than as the products of a large number of independent Bernoulli trials...


Goal: devise mathematics to ``capture the intuitive notion of randomness.''

Compare logicians' attempts to capture the intuitive notion of logical consequence.

 Howson and Urbach (1993: 324) that ‘it seems highly doubtful that there is anything like a unique notion of randomness there to be explicated’.

Intuitions about randomness:

\begin{itemize}
\item a property of a sequence
\item indeterminism
\item epistemic randomness
\end{itemize}


See also https://www.cs.auckland.ac.nz/~chaitin/sciamer.html

%%%%%%%%%%%%%%%%%%%%%%%%
\chapter{Infinity and the Limit Concept}

%%%%%%%%%%%%%%%%%%%%%%%%
\chapter{The Function Ladder: Differentiation and Integration}

\begin{remark}
  Why this?  Mainly just as a notational convenience.  Even if we target an
  audience with minimal math, it is useful to have $\int$ (or
  $\overset{+}{\lambda}$) to indicate the area under a curve.  And
  that is critical to the concept of probability of a continuous
  random variable, which is one of the main concepts we want to get
  across.

  Furthermore the basic concepts are not that difficult:
  differentiation as the limit of a ratio of differences, integration
  is the limit of a sum of products.  The details of how one might
  actually do this may be daunting for many readers, but the basic
  concepts are quite simple and intuitive.  Especially with lambda
  notation.
\end{remark}

\begin{remark}
  Furthermore there is an aesthetic component: symmetry.
  Differentiation and integration as relations among functions;
  comparable to the relation between a function and its inverse.

  Pedagology usually presents differentiation in terms of physical
  motion, instantaneous rate of change, etc.  Integration is presented
  as a matter of finding area under a curve.  But neither of these
  really capture what's at stake; this is obvious if you ask what the
  area under a curve has to do with probability.  Physical metaphors
  are not necessary to acquire a solid intuitions of differentiation
  and integration.  A (perhaps) better approach is to focus on the
  symmetrical relations involved.  Differentiation and integration as
  higher-order functions that map other functions to their ``natural''
  co-functions or counterparts.  So we start by just asking, e.g. what
  is the relation between \(x^2\) and \(2x\), or \(x^3\) and \(3x^2\).
  We can easily see a pattern involving coefficients and exponents;
  the derivatives and integrals account for these.

  Avoid ``rate of change'' talk at all costs.  Mathematics is static;
  there is no change.  So if \(2x\) is the derivative of \(x^2\) we
  need a way to express the intuition involved without appealing to a
  concept of change or motion.  Just: ratio, a ratio function of fixed
  form but different value at each x.  Not rate of change, but ratio
  of ``sizes''.  We can change our focus of attention to different
  values of x and y, but the functions themselves do not do this; they
  are fixed, unmoving, rigid.  So its like looking at a building and
  comparing its height to its width.

  But let's face it: there's no way to eliminate intuitions of
  rate-of-change.  Going from ratio of magnitudes to rate of change is
  natural.

  [An alternative perspective: (organic) growth.  A function grows;
    its derivative says how fast it grows, and its integral says how
    much ``mass'' it accumulates.]

  [Another alternative: behavior.  Functions ``behave'' differently at
    different places. etc.]

  If \(f(x)\) tells us where we are - how far along toward the end -
  then \(f'(x)\) tells us how fast we're moving.  Reverse perspective,
  and if \(f(x)\) tells us where we are, then \(\int f(x)\) tells us
  how much ground we've covered from the beginning.  The former is
  about rate, the latter about accumulation.

  The beauty of it is the symmetry.  If we have \(f\) and \(g\), where
  \(f\) is the derivative of \(g\), then \(g\) is the integral of
  \(f\).  By looking at \(f\) we can see how fast \(g\) is moving; by
  looking at \(g\), we can see how much ground \(f\) has covered.  The
  location of \(f\) describes the rate of \(g\); the location of \(g\)
  describes the accumulation of \(f\).

  If we think of \(f\) and \(g\) as two distinct moving bodies, then
  this fundamental relation between differentiation of the one and
  integration of the other - call it the conversion relation, by
  analogy with inverse relation - links their rates.  They move at
  different rates and thus cover different distances over the same
  ``time'', but those rates and distances are related.  Position,
  rate, and cumulative distance are the three things related by this
  conversion relation.  Corresponding to three kinds of question:
  where is it? how fast is it moving? and how far has it traveled?
  The function itself answers the first question, its derivative
  answers the second, and its integral answers the third.

  Note that there are two symmetries here.  One a kind of inverse
  relation between two functions, derivative and anti-derivative, and
  the other a kind of transitive relation between a function, its
  derivative and its anti-derivative.

  This exposes a fundamental difference between the notion of inverse
  function and the more special kind of inverse relation involved in
  differentiation and integration.  The plain old inverse relation
  only involves two functions; there is no transitive relation to some
  third function.  In other words, the inverse relationship is
  strictly symmetric but not transitive, and the ``conversion''
  relation is both.  You can go up and down the ladder.  But the
  ladder is well-founded; you can go up (integrate) indefinitely, but
  you always hit bottom going down.  More concretely: rates of rates
  of rates ... eventually end up at zero: no change at the nth degree
  rate.  But accumulation of accumulation of accumulation ... just
  keeps going.  (Since each ``rung'' in the upward ladder covers some
  ground.)

  Example: there are infinitely many ``rungs'' in the integration
  ladder of \(f(x)=x\), but going the other way always eventually
  bottoms out in zero.

  Bottom line: fundamental theorem of calculus expresses a transitive
  relation.

  But this means that each (well-behaved) function is actually, and
  essentially, part of a chain of functions.  The relation between a
  function, its derivative, and its integral is intrinsic.  So we
  cannot treat such functions as isolated individuals; rather they are
  akin to points on a continuum.

  Compare this to the exponential ladder.  Taking the nth derivative
  (integral) akin to taking the nth power (root).  Compare moments
  about the mean.

  Recursion?

  Fun, but how relevant to stats?  Maybe it will give us intuitions
  about distribs, etc.  Derivatives tell us something about their
  original functions, etc.  Moments about the mean tell us something
  about the original population. etc.  Mathematical techniques of
  using derived things to reveal information about basic things.

\end{remark}

%%%%%%%%%%%%%%%%%%%%%%%%%%%%%%%%%%%%%%%%%%%%%%%%%%%%%%%%%%%%%%%%
\chapter{Measure}


%%%%%%%%%%%%%%%%%%%%%%%%
\chapter{Thinking Exponentially (and Logarithmically)}

\begin{remark}
  Task 1: convince reader that exp and log, power and root, are
  special.  Task 2: Provide intuitive, clear, simple explanation.
  Task three: show relation to stats thinking.
\end{remark}

Learning to think in terms of exponents and logarithms is one of the
keys to learning to think statistically.  Fortunately it's not too
difficult, and it happens to be fascinating.

Getting from here to there.  We normally think linearly.  The number
line is treated as additive; to get from $a$ to $b$ on the number
line, you add the (possibly negative) difference between the two: $a +
(a-b) = b$.

Exponentiation offers a different way of thinking about the relation
between two numbers.  Instead of viewing their difference as spatial
distance ($a-b$), we think of it in terms of difference in growth.  So
the difference between, say, $3$ and $9$ is not $6 (= |3-9|)$, but $2
(= log_3\ 9)$.  Here we take $2$ as the number of (natural) growth
cycles it takes for $3$ to turn into $9$: $3^2 = 9$.

Why \textit{growth}?  Because exponentiation is self-contained, so to
speak.  Getting from $3$ to $9$ linearly involves adding something
external ($6$) to $3$.  But $3^2$ does not involve any external
``factor'' in this way (other than the multiplication operation); the
difference between $3$ and $9$ is construed in terms of $3$ alone;
``raising $3$ to a power'' is construed as an operation involving $3$
alone.  The task is to find how many times this something must be done
for $3$ to ``turn into'' (rather than ``arrive at'') $9$.

Traditional terminology (now archaic) captured this; as late as the
19th century, a common term for exponentiation was ``involution'',
meaning something like ``turning in to itself''.

Note that conventional terminology, being derived from Greek geometry
($3^2$ means ``three \textit{squared}''), is misleading.
Exponentiation has nothing essential to do with geometry.  (And note
that the Greeks did not have a genuine concept of exponentiation; they
never came up with the kind of algebraic thinking involved in finding
powers and roots.)  The Arabic-speaking mathematician who is credited
with inventing algebra in something like the form we know it today
(Al-Khawarizmi) did not use the term ``squared''; his (Arabic) word
for what we call a squared quantity was \textit{m\^{a}l}, meaning
``cattle, stock, wealth''.  Furthermore, the Arabic term for
multiplication was also conceptually distinct from the notion of
repeated ``folding'' (``multiply'' comes from the latin
\textit{multiplicare}, combining \textit{multi} ``many times'' and
\textit{plicare}, from \textit{plex} ``fold'').  The Arabic term was
\textit{darb}, ``strike'', and to form a \textit{m\^{a}l}, one
``strikes'' a number \textit{in itself}.  The origins of this usage
are not known, but note that the same term is used in minting
(``strike a coin'') and husbandry (``strike'' was used to refer to
copulation of e.g. livestock).  So historically, exponentiation was
did not emerge from either arithmetic nor geometry; rather, it emerged
as a distinctive concept.

But exponentiation is also conceptually distinct even (or especially)
in modern mathematics.  A satisfactory definition of exponentiation
requires some fairly advanced calculus; the simple concept of
something multiplied by itself is not sufficient.

It even shows up in very practical matters.  Calculation of compound
interest turns out to be intimately related to exponentiation, for
example.

\begin{remark}
  Story of $e$ as discovery of a mathematically satisfying account of
  exponentiation.  Exponentiation defined in terms of logarithms,
  rather than the other way around.
\end{remark}

The critical point of all this is that we should think of
exponentiation as a special kind of operation, rather than as
something derived from arithmetic (multiplication).  Fortunately this
is not particularly difficult, but it does require a basic change in
perspective.

The payoff will come when we consider probability distributions, in
which exponentiation plays a critical role.

\begin{remark}
  Also logarithmic scales, logit, etc.  Lots of places where logs and
  exponents are critical.
\end{remark}

\begin{remark}
  Exponentiation as involution.  Inflection points.  Symmetry.
  Constructing the numbers from exponentiation ($\lim_{x \to 0}e^x =
  0$).
\end{remark}

\begin{remark}
  Powers and roots v. exponentiation.  Difference between ${f(x) =
    x^a}$ (powers) and ${f(x)= a^x}$ (exponentiation).
\end{remark}

Sequences: any term of the following sequences can be used to form a
function, e.g. $f(x) = x^2, g(x) = 2^x$.

{\setstretch{1.25}
  \begin{alignat}{2}
    0^x, 1^x, 2^x,..,e^x,\ldots,n^x & \quad \textnormal{Power sequence?}
    x^0, x^1, x^2,..,x^e,\ldots,x^n & \quad \textnormal{Exponential (geometric) sequence} \\
  \end{alignat}
}

\section{Powers and Roots}

\begin{remark}
  Powers and roots as ``phases'' (``poles'', polarity?) of
  exponentiation, with $1$ as the ``inflection point''.  But we need a
  different term, ``inflection point'' should be reserved for changes
  in the derivative of a curve.  We want something that indicates a
  phase shift, where powers switch to roots and vice-versa.  ``Phase''
  by analogy to phases of matter (solid, liquid, gas), not of cycles
  (waves).  I.e. qualitative change, not cyclic orientation.  Critical
  point? (Freezing point, melting point, etc.)
\end{remark}

This can best be explained by example.

Let's review some basic rules:

{\setstretch{1.25}
  \begin{alignat}{2}
    a^{b} &= a\cdot a\cdots a \quad \textnormal{(b times)} \\
    a^{-b} &= \frac{1}{b} \\
    a^{1/b} &= \sqrt[b]{a} \\
    a^{b/c} &= \sqrt[c]{a^b}
  \end{alignat}
}

First powers.  Conventionally, the expression $a^b$ tells us to
multiply $a$ by itself $b$ times (that is, $b-1$ multipication
operations involving $b$ terms $1$).  This makes perfect sense, if $b$
is a whole number and $b>1$.  But what about, say, $a^{1.5}$?  Or even
worse, $a^\pi$?  As it happens, it is fairly easy (but perhaps not
trivial) to define $a^b$ for all rational $b$ (like $1.5$), but
defining it for irrational $b$ (like $\pi$) is another matter.
Explaining how this is done is beyond the scope of this paper, so for
our purposes let's just assume that $a^b$ is defined for all real $b$.

So $a^{1.5}$ and $a^\pi$ are defined, but what does it mean to
multiple $a$ by itself $1.5$ or $\pi$ times?  Not very intuitive.  One
strike against the conventional explanation of exponentiation in terms
of multiplication.

For now, though, let's stick with the multiplication idea. The point
here is that we can treat $a^b$ as a problem to be solved, or a task
to be accomplished: find $c$ such that $a^b = c$.  Let's look at the
structure of the task.  We are given two operands, $a$ and $b$, and an
operation (exponentiation); the task is to find out where these lead
us.  In other words, we do not know where we are going to end up, but
we know how to get there: by raising $a$ to the $b$th power.  In
brief, our task is to reach the destination, given the means of
getting there.

Now consider roots.  The expression $\sqrt[b]{a}$, unlike $a^b$, tells
us where we are to end up, but does not tell us how to get there.
``Find the $b$th power of $a$ means ``perform the exponent operation
$b$ times on $a$; it tells us what to do; but ``find the square root
of $a$'' does not mean ``perform the square root operation on $a$'';
it tells us what the destination is, but not what we need to do to get
there.  In other words, $\sqrt[b]{a}$ denotes a value related to $a$
and $b$, not an operation that uses $a$ and $b$; in contrast, $a^b$,
although it indirectly denotes the destination value, directly denotes
the operation to use to get there.

\begin{remark}
  Improve this.  Both can be viewed as denoting either a value, a
  process, a device, or all of the above.  Why not think of
  $\sqrt[b]{a}$ as denoting an operation?  Mainly because there is no
  such operation, or at least it isn't normally understood in that
  way.  Clearly in contrast to $a^b$.
\end{remark}

\begin{remark}
  Correction: $a^b$ only \textit{seems} to tell us what to do, because
  it does so in the case of integer exponents.  But in the case of
  real exponents, it precisely does \textit{not} tell us what to do.
  But this is evidence of the specialness of exponentiation.  The
  arithmetic operators do tell us what to do; that is how they are
  defined.  But we \textit{cannot} define exp and root in this way.
  In fact the best we can do is find a procedure for approximating the
  value.

  Alternatively: arithmetic ops defined via effective procedures.  Exp
  ops defined in terms of meaning, not procedure.  There may be many
  distinct procedures that can be used to solve them.

  This is obvious in the case of square root.  Easy to define, but I
  would guess even relatively well-educated people would not know how
  to go about computing a square root by hand.  There are dozens of
  methods for computing square roots, all of which (?) only
  approximate the answer.

  Ditto for the trig funcs.  Easy to provide a geometric definition,
  hard to say how to compute.  Well, semi-hard, not as hard as exp and
  log.

\end{remark}


\begin{remark}
  Put this in a graph: Here's another symmetry: given $b>1$, as $b$
  increases, $a^b$ decreases if $0<|a|<1$, but $a^b$ increases if
  $|a|>1$; if $a=0$, then $a^b=0$ for all $b$, and if $a=1$, then
  $a^b$ = 1 for all $b$.  So here $a=1$ is a critical point where
  $a^b$ changes direction, so to speak, and $a=0$ is a critical point
  - call it a ``constant point'' where $a^b$ is constant.

\end{remark}



\section{Exponentiation and Co-exponentiation}

%%%%%%%%%%%%%%%%%%%%%%%%
\chapter{Random Variables}

Random selection devices

%%%%%%%%
\chapter{Functions}

%%%%%%%%
\section{First- and Second-order Functions}

\begin{remark}
Family of functions/curves: function and meta-function.  $y=e^x$ is a
function; $y=e^{\alpha\ x}$ is a meta-function.  Better: second-order function.

Use lambda abstraction to demonstrate the difference: partial
application of a second-order function yields a first-order function.

Terminology: apply higher-order functions to \textit{parameters},
first-order functions to \textit{arguments}.
\end{remark}

%%%%%%%%
\section{Equations, Graphs, and Curves}

\begin{remark}
  Ideally, one would be able to instantly visualize the curve upon
  seeing the equation.  But there are many many functions for which
  this is not so easy.  But with a little practice it becomes
  relatively easy to know the basic shape of a curve from a glance at
  its equation.  So one purpose here is training in the art of seeing
  the curve in the equation.  Example: once you understand the
  relation between functions of $e$ and their curves, such as the
  curves of $e^x$ and $e^{-x}$, then it becomes relatively easy to see
  what shape the curve of the Gaussian PDF ought to have.

  Functions of the form $e^n$ are particularly common, where $n$
  itself can be any sort of expression, e.g. $e^{-(x/\lambda)^k}$
  (Weibull pdf).

  Mastering the shape of an equation really means mastering the shapes
  of a family of equations.

  We can show quite clearly what shape a family of functions has by
  using animation to show what happens as the parameters vary.  This
  will both expose the general shape of the (meta) function, and the
  role of the parameters.
\end{remark}

\begin{remark}
  A second critical point is that the parameters of a meta-function
\end{remark}

%%%%%%%%
\subsection{Characteristics of curves}

Kurtosis, skew, scedasticity - fancy Greek terms for sharpness, skew,
and scatter.  But useful for classifying curves by shape.

Thin/fat tails.
\url{http://en.wikipedia.org/wiki/Fat-tailed_distribution}.
E.g. Cauchy distrib,
\href{http://en.wikipedia.org/wiki/Stable_distributions}{stable
  distribs} (except the normal)



%%%%%%%%%%%%%%%%%%%%%%%%%%%%%%%%%%%%%%%%%%%%%%%%%%%%%%%%%%%%%%%%
%%%%%%%%%%%%%%%%%%%%%%%%%%%%%%%%%%%%%%%%%%%%%%%%%%%%%%%%%%%%%%%%
\part{Computation}

The concept of computation that emerged in the mid-19th century and
crystallized in the 1930s is a fundamentally new concept.  It is as
basic as math and logic; it effectively extends the classic quadrivium
(geometry, arithmetic, astronomy, and music.)

\begin{remark}
  Why important for intro to stats?  Mainly for historical reasons;
  both effective procedure and axiom of choice emerged at roughly the
  same time.  Also, an understanding of effective proc sharpens
  understanding of choice and randomness.

  Practically, the emergence of statistics in the 20th century as the
  lingua-franca of empirical science was heavily dependent on
  computation technologies.  For example, today Bayesian statistics is
  all the rage; this only became possible with the advent of powerful
  but inexpensive computing devices.  So in practice learning
  statistics means learning computation techniques.  Or, learning to
  think statistically requires skill in thinking computationally; in
  practice this means learning how to use statistical software
  packages.  Unfortunately, one can learn to use such packages without
  understanding; in fact that is a permanent hazard of statistics
  education.  Part of the purpose here is to integrate statistical and
  computational thinking, so that statistical software (languages)
  become conceptually transparent.

  Example: R's idiosyncratic terminology, such as ``mode'' for
  ``type'', ``factor'' for ``category'', etc.  Understanding
  underlying computational concepts such as type allows one to ``see
  through'' some of the ad-hoc idiosyncrasies of particular
  statistical software languages to the underlying statistical
  concepts.
\end{remark}


%%%%%%%%%%%%%%%%%%%%%%%%
\chapter{Computability and Decideability}

\section{Effective Procedure}


%%%%%%%%%%%%%%%%%%%%%%%%%%%%%%%%%%%%%%%%%%%%%%%%%%%%%%%%%%%%%%%%
\chapter{Computation and Proof}

On the amazing convergence of proof theory and computability theory.



%%%%%%%%%%%%%%%%%%%%%%%%%%%%%%%%%%%%%%%%%%%%%%%%%%%%%%%%%%%%%%%%
\part{Philosophy}

\begin{remark}
  Relevance of philosophy to stats?  Bridge between mathematics and world.
\end{remark}

Radical empricism: data first, then theory.

Rationalism: theory, then data.

%%%%%%%%%%%%%%%%%%%%%%%%%%%%%%%%%%%%%%%%%%%%%%%%%%%%%%%%%%%%%%%%
\chapter{Causality}

\begin{remark}
  Factor analysis - factor = causal factor
\end{remark}

%%%%%%%%%%%%%%%%%%%%%%%%%%%%%%%%%%%%%%%%%%%%%%%%%%%%%%%%%%%%%%%%
%%%%%%%%%%%%%%%%%%%%%%%%%%%%%%%%%%%%%%%%%%%%%%%%%%%%%%%%%%%%%%%%
\part{Science}

Science originally ``natural philosophy''.

\begin{remark}
  Science and scientism, pseudo-science, cargo-cult science.
\end{remark}

\begin{remark}
  Scientific method - ties philosophy to experience.
\end{remark}

\begin{remark}
  On the relationship between mathematics and the world.  Involving
  fundamental philosophical ideas, also pragmatics of \textit{doing}
  ``science''.

  The fundamental issue is what sorts of claims can statistics make
  about the world, how should we take them, etc.  The goal here is of
  course clarity.

  This is the natural place for a historical perspective, as well.
  How did statistical practice (and theory) evolve?
\end{remark}

\begin{remark}
  Stress: empiricism as a philosophy did not pan out - we are not mere
  observers of given data.
\end{remark}

Paradigmatic cases?  Astronomical observations; invention/discovery of
quantifiable temperature, psychometrics, etc.

%%%%%%%%%%%%%%%%%%%%%%%%
\chapter{Measurement and Error}

\begin{remark}
  The previous section discuss the mathematical concept of measure.
  This section discusses theories of empirical measure\textit{ment}.
\end{remark}

%%%%%%%%%%%%%%%%%%%%%%%%%%%%%%%%%%%%%%%%%%%%%%%%%%%%%%%%%%%%%%%%
%%%%%%%%%%%%%%%%%%%%%%%%%%%%%%%%%%%%%%%%%%%%%%%%%%%%%%%%%%%%%%%%
\part{Probability}

%%%%%%%%%%%%%%%%%%%%%%%%%%%%%%%%%%%%%%%%%%%%%%%%%%%%%%%%%%%%%%%%
\chapter{Probability}

%%%%%%%%%%%%%%%%%%%%%%%%%%%%%%%%
\section{Measure Theory}

%%%%%%%%%%%%%%%%%%%%%%%%%%%%%%%%
\section{Probability Measures}

%%%%%%%%
\subsection{Joint Probabilities}

%%%%%%%%
\subsection{Conditional Probabilities}

%%%%%%%%%%%%%%%%%%%%%%%%%%%%%%%%
\section{Likelihood}

``In statistics, a likelihood function (often simply the likelihood)
is a function of the parameters of a statistical model. The likelihood
of a set of parameter values, θ, given outcomes x, is equal to the
probability of those observed outcomes given those parameter values,
that is $\mathcal{L}(\theta |x) = P(x | \theta)$
(\url{http://en.wikipedia.org/wiki/Likelihood_function})

%%%%%%%%%%%%%%%%%%%%%%%%%%%%%%%%%%%%%%%%%%%%%%%%%%%%%%%%%%%%%%%%
\chapter{Random Variables}

%%%%%%%%%%%%%%%%%%%%%%%%%%%%%%%%
\section{Discrete Rvs}

%%%%%%%%%%%%%%%%%%%%%%%%%%%%%%%%
\section{Continuous Rvs}

%%%%%%%%%%%%%%%%%%%%%%%%%%%%%%%%
\section{Algebra of Rvs}

%%%%%%%%%%%%%%%%%%%%%%%%%%%%%%%%
\section{Moments}

``In probability theory and statistics, a central moment is a moment of a probability distribution of a random variable about the random variable's mean; that is, it is the expected value of a specified integer power of the deviation of the random variable from the mean. The various moments form one set of values by which the properties of a probability distribution can be usefully characterised. Central moments are used in preference to ordinary moments, computed in terms of deviations from the mean instead of from the zero, because the higher-order central moments relate only to the spread and shape of the distribution, rather than also to its location.''\url{http://en.wikipedia.org/wiki/Moment_about_the_mean}

\begin{remark}
  NB analogy between moment of a distrib (a function) and derivative
  of a function.  Nth moment, Nth derivative, etc.

  Moment: expected value of nth involution of deviation (so we could
  generalize, so $y = log_{\bar{X}-\mu} m$ for any m, yielding $m$ as
  the $y^{th}$ moment.

  Compare finding nth derivative, or, given a function, finding the
  nth anti-derivative (integral).

  So why moments?  I suspect they're like derivatives: just as
  derivatives tell us something about the original function, moments
  tell us something about the original distribution (i.e. the random
  var, which is a function).  Or so I would expect, based on nothing
  more than a principle of symmetry.
\end{remark}

%%%%%%%%%%%%%%%%%%%%%%%%%%%%%%%%
\subsection{Expected Value}

%%%%%%%%%%%%%%%%%%%%%%%%%%%%%%%%
\subsection{Variance}

%%%%%%%%%%%%%%%%%%%%%%%%%%%%%%%%
\subsection{Skew}

%%%%%%%%%%%%%%%%%%%%%%%%
\chapter{Probability Distributions}

\url{http://en.wikipedia.org/wiki/Probability_distribution}


\begin{remark}
  Stress: connection between concepts of random var and prob. distrib
\end{remark}

A probability distribution is just a function, or rather a family of
functions expressed as parameterized equations.

\begin{remark}
  Family of functions is the critical idea.  Goal is to pick the best
  family, then the best function from the family.
\end{remark}

Ways of specifying probability distributions:

%%%%%%%%%%%%%%%%%%%%%%%%%%%%%%%%
\section{Probability Mass Function}


%%%%%%%%%%%%%%%%%%%%%%%%%%%%%%%%
\section{Probability Density Function}

%%%%%%%%%%%%%%%%%%%%%%%%%%%%%%%%
\section{Cumulative Distribution Function}

%%%%%%%%%%%%%%%%%%%%%%%%%%%%%%%%
\section{Characteristic Function}

%%%%%%%%%%%%%%%%%%%%%%%%%%%%%%%%
\section{Moment-generating Function}

``In probability theory and statistics, the moment-generating function of a random variable is an alternative specification of its probability distribution. Thus, it provides the basis of an alternative route to analytical results compared with working directly with probability density functions or cumulative distribution functions. There are particularly simple results for the moment-generating functions of distributions defined by the weighted sums of random variables. Note, however, that not all random variables have moment-generating functions.''\url{http://en.wikipedia.org/wiki/Moment-generating_function}

%%%%%%%%%%%%%%%%%%%%%%%%%%%%%%%%
\section{Hazard Function}

I.e. failure rate.

``Failure rate is the frequency with which an engineered system or
component fails, expressed, for example, in failures per hour. It is
often denoted by the Greek letter $\lambda$ (lambda) and is important in
reliability engineering.''
\url{http://en.wikipedia.org/wiki/Hazard_function#hazard_function}

``Calculating the failure rate for ever smaller intervals of time,
results in the hazard function (also called hazard rate), $h(t)$. This
becomes the instantaneous failure rate as $t$ tends
to zero...A continuous failure rate depends on the existence of a
failure distribution, $F(t)$, which is a cumulative
distribution function that describes the probability of failure (at
least) up to and including time t...''
\url{http://en.wikipedia.org/wiki/Hazard_function#hazard_function}

See \href{http://en.wikipedia.org/wiki/Bathtub_curve}{Bathtub curve}

See \href{http://en.wikipedia.org/wiki/Weibull_chart}{Weibull distrib}

%%%%%%%%%%%%%%%%%%%%%%%%%%%%%%%%
\chapter{Well-known Distributions}

%%%%%%%%%%%%%%%%%%%%%%%%%%%%%%%%
\section{Discrete}

%%%%%%%%
\subsection{Degenerate Distrib}

``The degenerate distribution at x0, where X is certain to take the
value x0. This does not look random, but it satisfies the definition
of random variable. This is useful because it puts deterministic
variables and random variables in the same
formalism.''\url{http://en.wikipedia.org/wiki/Degenerate_distribution}

%%%%%%%%%%%%%%%%%%%%%%%%%%%%%%%%
\section{Continuous}
\subsection{Gauss}

\begin{equation}
  \frac{1}{\sigma\sqrt{2\pi}}e^{-\frac{(x-\mu)^2}{2\sigma^2}}
\end{equation}

But:

\begin{equation}
  e^{-\frac{(x-\mu)^2}{2\sigma^2}} = \frac{1}{e^\frac{(x-\mu)^2}{2\sigma^2}} =
  \frac{1}{\sqrt{e^{\frac{(x-\mu)^2}{\sigma^2}}}} =
  \frac{1}{\sqrt{e^{\frac{x-\mu}{\sigma}}}}
\end{equation}

Now set $\mu=0, \sigma=1$.  Then

\begin{equation}
  \frac{1}{\sigma\sqrt{2\pi}} = \frac{1}{\sqrt{2\pi}} = 0.398942
\end{equation}

and

\begin{equation}
  e^{-\frac{(x-\mu)^2}{2\sigma^2}} =
  e^{-\frac{x^2}{2}} =
  \frac{1}{e^\frac{x^2}{2}}
\end{equation}

Then if x=0, we have

\begin{equation}
  e^{-\frac{(x-\mu)^2}{2\sigma^2}} =
  \frac{1}{e^\frac{0}{2}} =
  \frac{1}{e^0} = \frac{1}{1} = 1
\end{equation}

so when $\mu=0, \sigma=1, x=0$,

\begin{equation}
  \frac{1}{\sigma\sqrt{2\pi}}e^{-\frac{(x-\mu)^2}{2\sigma^2}} = 0.398942
\end{equation}

As $x$ grows in either direction from zero, $e^{-\frac{(x-\mu)^2}{2\sigma^2}}$ ...

The turning point is set by $\sigma^2$, since that determines when
$\frac{(x-\mu)^2}{2\sigma^2}$ pivots about $1$.  As $x$ gets larger beyond that point,
$e^{-\frac{(x-\mu)^2}{2\sigma^2}}$ gets smaller.

Main point: $e^{-\frac{(x-\mu)^2}{2\sigma^2}}$ never exceeds $1$,
since the exponent is never less than zero, and $e^0=1$. Then for all
$x>0$, $e^x$ gets bigger, so $\frac{1}{e^x}$ will always be $<1$.  So
the product of the two factors will always be less than $0.398942...$
The bell shape comes from the varying rate of change of
$e^{-\frac{(x-\mu)^2}{2\sigma^2}}$, which reflects the growth curve of
$\frac{1}{e^x}$, which declines relatively rapidly near zero, but
flattens out as $x$ increases.

The key is in the derivatives:

{\setstretch{1.5}
  \begin{alignat}{2}
  \overline{\lambda}x.{e^{x}} &= e^{x} \\
  \overline{\lambda}x.{e^{-x}} &= -e^{-x} \\
  \overline{\lambda}x.x^{\sigma x} &= \sigma x^{\sigma x} \\
  \overline{\lambda}x.x^{-\sigma x} &= -\sigma x^{-\sigma x}
  \end{alignat}
}

So as $\sigma$ varies, so does the flattness of the curve.

\begin{remark}
  Inflection points: in addition to the max and min, we have the
  points at which the second derivative switches from pos. to neg.  In
  other words, the point at which the normal curve starts to flatten.
  This happens when $x=\mu\pm\sigma$.  This makes intuitive sense
  since it is the point at which $\dfrac{(x-\mu)^2}{2\sigma^2}$ is $1$
  (assuming $\mu=0, \sigma=1$), which makes it the place where the
  exponent switches from power to root.
\end{remark}

\begin{remark}
  Plot $e^{nx}$ as $n$ varies.  As $n$ decreases, the graph becomes
  flatter; as it increases, it begins to approximate a sharp L shape.
  This is the effect of varying $\sigma$; as it decreases, the
  exponent increases.  Better: plot $e^{\frac{x}{n}}$.
\end{remark}

All of which goes to show, that the normal curve is a kind of
exponential curve, which makes it a kind of growth curve.  It
describes the way error grows as more observations are made.


%%%%%%%%%%%%%%%%%%%%%%%%
\chapter{Sampling}

\begin{remark}
  Distinguish between mathematical and empirical usage.  Here we talk
  of carrier sets instead of populations, elements instead of sample
  units, values instead of measurements, etc.

  If we think of sampling clearly, in terms of sets, sequences,
  etc. then the inferential stuff becomes intuitively clear.
\end{remark}

\section{Samples}

A sample is a sequence of elements drawn from the carrier set (population).

%%%%%%%%%%%%%%%%%%%%%%%%%%%%%%%%
\section{Sampling Continuous Random Vars}

Discrete v. continuous sampling?  A continuous sample would have to
select ranges rather than elements.  But often we want discrete
sampling of a continuous var; e.g. distance from bullseye.  This would
give us a multiset, but what would the elements and multiplicities be?
E.g. {3[.25]} would mean three events of striking within .25 of the
center.

\section{Sampling Sequences}

\begin{remark}
  Or: sample sequences.  Sequence stuff from \ref{subs:sequences}, from the perspective of sampling.
\end{remark}

\section{The Sampling Pool}

Or sample pool

The sampling pool is the union of the samples in the sampling sequence.

\section{Discrete}

\begin{remark}
    What we're going to want is counts of number of possible samples,
    number of unique samples (combinations), distribution of samples
    when the sampling sequence is long, distribution of elements in
    the sampling pool.  Etc.
\end{remark}

\chapter{Multivariate stuff}


\href{Exploratory multivariate analysis by example using R}{http://pi.lib.uchicago.edu/1001/cat/bib/8265780}



%%%%%%%%%%%%%%%%%%%%%%%%%%%%%%%%%%%%%%%%%%%%%%%%%%%%%%%%%%%%%%%%
\part{Statistics}

\begin{remark}
  What distinguishes stats from probability?  Probability seems to
  have everything stats has: expected value (for mean), etc.  Even
  statistical inference is based entirely on a result from
  probability, the Central Limit Theorem.  So why not call it all
  probability?

Statistics measures something else
  - location (central tendency), spread (deviations), etc.
\end{remark}

%%%%%%%%%%%%%%%%%%%%%%%%%%%%%%%%%%%%%%%%%%%%%%%%%%%%%%%%%%%%%%%%
\chapter{Descriptive Statistics}

%%%%%%%%%%%%%%%%%%%%%%%%%%%%%%%%
\section{Location}

%%%%%%%%
\subsection{Mean}
\label{subs:mean}

\begin{ednote}
  key terms: uniform distrib; equity; equitable distrib; allocation;
  parameterization of r.v.; nth degreee
  equation/function/mean/deviation; moments as \(n^{th}\)
  mean/deviation.
\end{ednote}

\newthought{Interpretation of the Arithmetic Mean}

\begin{ednote}
  \(\dfrac{\sum x}{n-1}\) is not the arithmetic mean; but it is a
  \emph{kind} of mean.  We could call it ``mean of randomization'',
  ``free mean'', or the like.

  We can treat the arithmetic mean as a description of a single
  particular distribution (set).  It can be interpreted in multiple
  ways, one per polynomial degree.  At degree zero, it is the value of
  the constant function; at degree 1, the midpoint of both the domain
  and range of a linear function; etc.

  By contrast, we think of the ``free mean'' not as a determinate
  description, but as a kind of generalized representation of a
  \emph{family} of samples (sets), all of which are ``equivalent up
  to'' cardinality and and sum.  ``Equivalent up to cardinality and
  sum'' is a fancy way of saying that each sample in the family has
  the same number of elements, which sum to the same total.  It
  follows that each sample has the same mean.  We could say
  ``equivalent up to mean'', except same mean does not imply same
  cardinality.

  In other words, given a specific sample set, there is an infinite
  number of different sample sets that ``look the same'' as the
  original set in that they have the same cardinality and the same sum
  (and thus the same mean).  The ``free mean'' describes that set; we
  can use it, so to speak, to generate specific samples, by random
  selection parameterized by the free mean.

  Another constraint: not only do we want our \(n-1\) choices to be
  random; we also want them to have the ``same amount'' of randomness.
  In other words, we want our randomization devices to be identical
  (cf. i.i.d.).  This maximizes total randomness.  Note that we could
  ``bias'' the randomness by giving each device a different range.

  There is a basic tradeoff here.  In order to maximize total
  randomness across \(n-1\) devices, we are forced to restrict the
  possible variance.  This means restricting the range allocated to
  each device.  If we were to allocate more than the free mean amount
  to the devices, then the possibility arises that the total could be
  ``used up'' before we get through all \(n-1\) devices.  In that case
  we would not have \(n-1\) free (random) choices.  So we compromise,
  by arranging things such that even if every choice selects the
  maximum possible value, there will always be ``enough'' choice left
  for the remaining devices, up to \(n-1\).  In this extreme scenario,
  the first \(n-1\) choices exhaust the entire total available,
  leaving zero for the final (\(n^{th}\)) element.  But that's ok,
  because that element is deterministic in every case; it is never
  freely chosen.

  So our compromise means that no value of our sample will exceed
  \(\dfrac{\sum x}{n-1}\).  This effectively means we limit the
  possible variance in our sample in order to guarantee maximal
  total randomness.

  Note that the family (=set) of samples determined by the free mean
  is not exhaustive; there is yet another infinity of equivalent
  samples that are not ``explainable'' by the free mean.  For example,
  consider the sample whose first element consumes the entire total,
  so that all remaining points get zero.  This counts as equivalent up
  to cardinality and sum, but it has only one degree of freedom.
  Once the first element exhausts the available quantity, the
  remaining points are completely determined: they get zero, and no
  random choice is involved.

  This suggests we should add something like ``degree of randomness''
  to our notion of ``equivalent up to''.  Samples determined by the
  free mean are equivalent to the original sample up to cardinality,
  sum, and (degree of) randomness.  But this doesn't quite work,
  since randomness is not a property of specific samples, but of the
  selection process.  And besides, it's not quite true that they are
  equivalent in this manner; all \(n\) points of the original were
  randomly selected, but only \(n-1\) points in the equivalent sample
  were.

  So perhaps a better terminology is: the ``free mean'' determines an
  infinite set of samples that are equivalent to the original ``up to
  cardinality and sum'', and maximizes total randomness.

  Critical point: the notion that the free mean determines samples is
  based on interpreting it as a parameter setting the range of random
  variables.  So the analogy is to a machine consisting of an array or
  chain of randomization devices whose output range is determined by a
  parameter setting, and the free mean serves as that parameter.
  ``Running'' the machine yields a maximally random sample equivalent
  up to the cardinality and sum to the original sample.
\end{ednote}

\begin{ednote}
  Begin with distinction between determinate and indeterminate values
  (function as deterministic ``process'' v. random process).  In our
  original set of values, each value construed as randomly generated.
  That's the critical difference between the statistical
  interpretation of arithmetic mean and its ordinary mathematical
  definition.
\end{ednote}

\begin{ednote}
  Something about concept (and history) of ``normal distribution of
  error'' and how it relates to concept of r.v.  Isn't it the
  motivation for the various interpretations of mean and variation
  described below?  We want to know which probability distrib provides
  the best ``model'' (= ``explanation''?) of the data.  NB:
  prob. dist. is a function, serves as deterministic mathematical
  model of randomness?
\end{ednote}

\begin{ednote}
  We start by ``selecting'' \(n\) random numbers; once they are
  selected, we find their sum \(\tau\).  The goal is to ``mimic'' the
  original selection - make the same number of selections that sum to
  the same total.  This would give us two distinct distributions with
  the same mean---two sets that are equivalent ``up to the mean''.
  Ideally every number in the second set would be selected at random,
  just as was the case in the first set.  But there is no way to
  guarantee that \(n\) numbers selected at random will sum to a
  particular number.  So we know from the beginning that at least one
  of our selections must not be random.  Specifically, the last
  selection is determined by the sum of the preceding selections.
  This follows from the equation \(x_1 +\cdots +x_n = \tau \Rightarrow
  x_n = \tau - x_1 +\cdots x_{n-1}\).  In other words, \(x_n\)
  represents the quantity remaining to be allocated after the first
  \(n-1\) numbers have been selected randomly, and it is completed
  determined: it is just the difference between \(\tau\) and the sum \(x_1 +\cdots x_{n-1}\).

  And so forth.  The key point is that the goal of ``mimicking'' the
  original distribution [sidenote{``Mimickery'' may not be the right
      metaphor.  What we're trying to do is find members of the
      equivalence class determined by the mean \(\mu\); that is,
      distributions whose mean is just \(\mu\).}] involves two
  sub-goals that are in tension: maximizing randomness, and summing to
  the predetermined total.  The task is to find the optimal
  combination of random and determinate values.  That solution is in
  terms of ``degrees of freedom'', which is the designated terminology
  for ``number of random choices involved''.  For a set of \(n\)
  values summing to \(\tau\), only \(n-1\) degrees of freedom are
  available, meaning that we can make a random selection only \(n-1\)
  times---the last number to be selected in order to make the sum
  total to \(\tau\) is completely determined by the preceding \(n-1\)
  random numbers.

  But then: why \(n-1\)?  Why not \(n-2\), or some other number?

  And also: how can we make sure that all \(n-1\) selections are
  indeed random?  One way in which this constraint could be violated
  is if the sum were to total \(\tau\) before we get to \(n-1\); say, on
  the \(n-2\) selection.  In that case, we would have not one but two
  values determined by preceding selections, at points \(n-1\) and
  \(n\).  So in such a case we would \emph{in fact} have fewer than
  \(n-1\) random numbers.  But the goal is to ensure that we have
  maximum randomness, or alternatively, that we have minimal
  determinateness, which means we are required to ensure that all
  \(n-1\) selections are random in fact.  Only the last (\(n\))
  selection is allowed to be determined by prior selections.
\end{ednote}

\begin{ednote}
    Why maximize randomness?  I think this is the link to the concept of
  ``normal distribution of error''.
\end{ednote}

The arithmetic mean of a set of \(n\) numbers can be thought of in
terms of redistribution of total across \(n\) ``sample points''.  The
most obvious way to think of this, as suggested by the formulaic
definition of the mean, is in terms of a constant function that
assigns an equal portion of the total to each sample point.

This means we think of the mean in terms of a \emph{counter-factual}
situation.  We interpret the original distribution across the sample
points as a \emph{fact}; it's what the world (i.e. the set) looks like
when the total is distributed according to some (unknown) rule.  Then
we can ask what the world would look like if the total were
distributed according to some other rule.  The arithmetic mean tells
us what things \emph{would} look like \emph{if} the total were
distributed according to a rule that assigns the same amount to each
sample point; in other words, if the rule of assignment were a
constant function.

There are two constraints on the redistribution.  One is the obvious
requirement that the sum total after redistribution must equal the sum
total before redistribution.  That is the point after all; we want to
know just how things would look if the \emph{same} total quantity were
differently distributed.

The other constraint is less obvious: it requires that the
redistribution be according to a \emph{simple} rule.  The ``rule''
must of course be a function; but that is not enough: the function
must be expressible by a single equation.  This rules out step
functions, which must be expressed by two or more equations.

The motivation for the second constraint is best illustrated by
example.  If all we want is redistribution according to rule, then we
can define any number of rules that assign a fixed quantity to each
sample point.  This \emph{seems} to be what we're doing with the
arithmetic mean (my reasons for saying ``seems'' will become obvious
in a moment).  On this view, each sample point gets exactly \(\mu\)
units.\sidenote{We will always use \(\mu\) to denote the mean; where
  necessary for disambiguation we'll add a subscript, e.g. \(\mu_k\).}
In mathematical terms, this means we are using the constant function
\(f(x) = \mu\) as our distribution rule.  Under this kind of rule, the
quantity assigned does not depend on the sample point; they all get
the same allotment.

But there are many other possibilities.  Indeed, we can effect a fixed
redistibution of this sort by using functions---that is, polynomical
equations---of any degree.  Recall that the degree of a polynomial in
\(x\) is defined as the largest exponent of \(x\) that appears in the
defining equation.  So the constant function \(f(x) = a\) has degree
\(0\); it can be interpreted as \(f(x) = ax^0 + 0 = 1\cdot a = a\).  A
function of degree \(1\) has the form of \(f(x) = ax^1 + b = ax + b\)
--- an ordinary linear equation, where the independent variable (in
this case, \(x\)) has an exponent of \(1\).\sidenote{Remember that the
  domain of our functions is the ordered set of integers \([1\upto
    n]\).  So the allotment assigned to any sample point would depend
  just on where it is in that ordered set.}  A function of degree
\(2\) has the form \(f(x) = ax^2 + bx + c\), and so forth.  The
symbols \(a, b, c,\ldots\) in such equations are called
\emph{parameters}; the idea is that such equations are
\emph{schematic} forms that \ determine a \emph{family} of equations
(functions), one for each assignment of fixed values to the
parameters.  In other words, \(f(x)=ax+b\) does \emph{not} define
a specific function; rather it expresses a general pattern or
abstraction to which an infinite number of specific functions conform.
Taking \(f(x)=ax+b\) as an example, by fixing \(a=3, b=7\) we
obtain (by substitution) the concrete equation \(f(x)=3x+7\);
fixing \(a=1, b=0\) we obtain the identity function \(f(x)=x\).  And
so forth.

Now back to our specific question: what would our set look like
\emph{if} we were to use a first degree distribution function based on
the arithmetic mean?  We are given the general form \(f(x)=ax+b\); the
task is to find values for the parameters \(a\) and \(b\) such that
the resulting concrete equation defines a suitable redistribution
function.\sidenote{Note that an implicit requirement is that \(a\neq
  0\), since \(f(x) = 0x + b = b\) defines a constant function.  (Or
  more accuratly, it defines the form of a family of such functions.)}

In this case, each sample point would be allocated a different
quantity, since the value of \(f(x)=ax+b\) depends on the value of
\(x\).  The graph of an equation of this form is a line.  More
specifically, it is a line whose slope is non-zero; this is another
way of saying it is not a horizontal line---a horizontal line is the
graph of a constant function.\sidenote{It also cannot be a vertical
  line, since the graph of a genuine function cannot be a vertical
  line.  Another way of looking at it is to observe that the infinite
  family of equations (functions) described by \(f(x)=ax+b\)
  corresponds to the infinite number of lines in the plane
  \emph{except} the (equally infinite) set of horizontal and vertical
  lines.}

Now there are an infinite number of first degree equations we could
use to redistribute our total.  Which one or ones do we want, and why?
Here we have two constraints.  One is that the total after
redistribution must equal the total before redistribution.  The other
is that the redistribution must range over the same number of sample
points as in the original set.  Technically this means we are placing
a constraint on the \emph{domain} of our function.  Normally, function
definitions of the form \(f(x)=ax+b\) assume implicitly that the
independent variable \(x\) ranges over the real numbers \R.  Since we
are interested in a discrete set of sample points\marginnote{TODO:
  address discrete v. continous at beginning}, this means we need to
place a \emph{domain restriction} on our function.  Symbolically we
can express this using the domain restriction operator\sidenote{This
  symbol is defined by the Z notation.}  \(\dres\) as follows: \(S_n
\dres f(x)=ax+b\), where \(S_n\) denotes a ``selection function'' that
selects \(n\) points from the domain of \(f\).  The whole equation
means, informally, ``the function whose domain is determined by
\(S_n\) and whose range is determined by \(f\)'', or alternatively
``the function defined by \(f(x)=ax+b\), with \(x\) constrained to
range over the set determined by \(S_n\)''.

Now the key question becomes: what is the definition of \(S_n\)?  And
more to the point, what motivates it?

\begin{ednote}
  NB: we could also use various functions for \(S_n\).  Why is a
  uniform linear distrib best?

  TODO: plot some examples.  Use \(n\) points from the unit interval
  with \(\mu=.5\).  Animate the interpretation of \(\mu\) as we move
  from \(f^0\) to \(f^3\).
\end{ednote}


\begin{ednote}
  This is domain restriction by rule.  So we actually have two rules.
  The domain restriction rule selects (that is, distributes) points
  from the domain, which determines the points for which we will
  compute values; and the ``redistribution rule'' is the function that
  assigns values to the selected points.

  We can express this conveniently by naming these two rules the
  ``selection function'' and the ``redistribution function''.  Both
  are deterministic; the former fixes the domain of the latter; the
  latter fixes the values assigned to the selected points.

  Explain the difference between a function and its values, and random
  variable ``observations''.  When we use counterfactual functions to
  interpret the arithmetic mean, we're moving from random variables to
  determinate functions.  But this means changing our domain as well;
  we replace ``taking'' \(n\) samples with evaluating functions over
  infinite domains, which obviously include more than \(n\) points.
  The mean comes out as characterizing the function is some critical
  way---as the value of a constant function, the midpoint of a linear
  function, and so forth.  In other words, such counterfactual
  thinking allows us to \emph{treat} random data \emph{as if} it were
  ``produced'' by a function.  Then our original taking of samples,
  which was ``observing'' the \emph{value} of a random variable \(n\)
  times, corresponds to sampling the domain of a function.  And to
  make this work, we move from \(n\) random samples to \(n\) points in
  the function domain distributed linearly.  I.e. we select \(n\)
  points from the domain such that the distance between any two
  adjacent points is the same.

\end{ednote}

 Intuitively, we are interested in the line that runs from
the minimum value of the original set to its maximum value.  By the
definition of the arithmetic mean, this interval from minimum to
maximum is guaranteed to contain the mean.  In fact, the arithmetic
mean of the original set will correspond exactly to the midpoint of
this line.



\newthought{But this is statistics}; we are not so interested in fixed
values.  We want to know not only how things look under a different
distribution rule; we also want to know how things might look under
distributions rules that have a \emph{random} component.

\begin{ednote}
  Having moved from the original (possibly) random distribution to
  counterfactual deterministic redistributions, we restore randomness
  by another reinterpretation.  Instead of thinking of our
  redistribution function as fixing values for domain points, we treat
  it as fixing the parameterization of random variables---as setting
  ``knobs'' on randomization devices.  Each point is construed as a
  randomization device.

  But this means we have moved back away from the function concept.
  We no longer compute the value of the redistribution function for
  each point in the restricted domain.  Instead we construct \(n\)
  randomization devices as parameterized random variables, where
  ``parameterization'' sets the range over which the r.v.s may vary.
  Then we ``observe'' each r.v. and sum the results.

  An obvious problem: how can we guarantee that our results will sum
  to the required total?  This is where degrees of freedom and the
  concept of bias enter the picture.  Degress of freedom always leave
  some points whose allocation is determined by the preceding
  allocations.  This means that once the randomizers have generated
  their values, we can compute how much ``stuff'' remains to be
  distributed and assign that determinately to the remaining points.
  So this mix of ``free'' and ``fixed'' points amounts to the optimal
  solution to the task of maximizing randomness while ensuring that
  the overall constraints are satisfied.

  A simple question: why bother with all this?  Why do we want to
  allocate a total across some set of randomization devices?  Possible
  answer: because this represents fitting the observed values to a
  probability distrib.  The real question for which we want an answer
  is: how would things look if we were to use the distribution
  functions defined by probability theory as our redistribution
  function?  And which of those distribution functions yield the
  ``best'' redistribution---the one that comes closest to matching the
  original data?
\end{ednote}


\begin{ednote}
  Order of exposition/illustration: start by making the entire total
  avaliable to each device.  Problem is that first point can use up
  the entire total; then all the remaining allocations are
  deteministically fixed.  One possible remedy: add clauses to the
  redistrib rule to the effect that allocation to point \(n\) depends
  on previous allocations; this is disallowed (why?).  Next try:
  allocate according to simple functions of degree n, as above.
  Difference is allocation as parameterization rather than fixing of
  values.  First try: linear alloc evenly distributed across all
  points.  This induces bias (why/how?).  Solution: degrees of
  freedom.  Distribute across \(n -\) degrees of freedom.  This makes
  those points random, and the remaining deterministic.  This is the
  optimal (only?) redistrib method that satisfies the basic
  requirements and maximizes randomization.
\end{ednote}

\begin{ednote}
  Need a good, simple, intuitive account of the technical concept of
  bias.
\end{ednote}

%%%%%%%%%%%%%%%%%%%%%%%%%%%%%%%%
\section{Spread}

%%%%%%%%%%%%%%%%%%%%%%%%%%%%%%%%
\section{Curve fitting}

aka Hypothesis Testing

%%%%%%%%%%%%%%%%%%%%%%%%%%%%%%%%
\section{Association}

%%%%%%%%%%%%%%%%%%%%%%%%%%%%%%%%%%%%%%%%%%%%%%%%%%%%%%%%%%%%%%%%
\chapter{Inferential Statistics}

%%%%%%%%%%%%%%%%%%%%%%%%%%%%%%%%
\section{From Sample to Population}

%%%%%%%%%%%%%%%%%%%%%%%%%%%%%%%%
\section{From Correlation and Causality}

%%%%%%%%
\subsection{From Manifest to Occult}

IOW, from observable to latent variables.



%%%%%%%%%%%%%%%%%%%%%%%%%%%%%%%%%%%%%%%%%%%%%%%%%%%%%%%%%%%%%%%%
\part{A Statistical Beastiary}

Specific statistics, distribs, tests, etc. - how they are used in practice.

%%%%%%%%%%%%%%%%%%%%%%%%%%%%%%%%
\chapter{Chi Squared}
\label{sect:chisquared}

To find the expected multiplicities, form the disjoint union of the marginals.

\begin{remark}
  Not sure ``disjoint union'' is the right term.  The idea is that we
  take the observed marginals, say A and X, and form the set of all
  pairs (a,x) by multiplying.  Effectively this means we take all
  observed pairs (a,\_) and (\_,x), ignore the second and first
  members, respectively, and then ``cross'' them.  So for every (a,\_)
  we get X copies of (a,x), one for each (\_,x).  The result has
  multiplicity of \(A\times X\).  I think this is disjoint union but
  I'm not sure.
\end{remark}

%%%%%%%%%%%%%%%%%%%%%%%%%%%%%%%%%%%%%%%%%%%%%%%%%%%%%%%%%%%%%%%%
\part{Causality}

%%%%%%%%%%%%%%%%%%%%%%%%%%%%%%%%%%%%%%%%%%%%%%%%%%%%%%%%%%%%%%%%
\chapter{Factor Analysis}

\url{http://psycnet.apa.org.proxy.uchicago.edu/index.cfm?fa=browsePB.chapters&pbid=10694}


\clearpage
\part{Appendices}
\appendix
\begin{appendices}
\chapter{Bibliography}
%% \addcontentsline{toc}{chapter}{Bibliography}
\bibliography{logic,math,stats%
%% ../bib/logic.bib%
%% ,../bib/math.bib%
%% ,../bib/stats.bib%
}
\bibliographystyle{plainnat}
%% \printbibliography[heading=none]
\end{appendices}

\end{document}
