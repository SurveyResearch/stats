\part{Description}

\begin{ednote}
  Description of data, not data as description of world - that's modeling
\end{ednote}

In principle, every datum must be associated with a complete
description that determines both its form and its meaning.

\begin{description}
\item [Syntax]
\item [Semantics]
\item [Type]  Same as syntax?
\item [Traceability] esp. given ability to rename, split, and
  generally slice and dice, traceability is critical.
\end{description}

%%%%%%%%%%%%%%%%%%%%%%%%%%%%%%%%%%%%%%%%%%%%%%%%%%%%%%%%%%%%%%%%
\chapter{Datum: description of types and individuals}

%%%%%%%%%%%%%%%%%%%%%%%%%%%%%%%%%%%%%%%%%%%%%%%%%%%%%%%%%%%%%%%%
\chapter{Data: description of data aggregates}

Descriptive statistics.  The technical statistical side of such
description is covered in detail in the companion volume.  Here we
cover ...?

%%%%%%%%%%%%%%%%%%%%%%%%%%%%%%%%%%%%%%%%%%%%%%%%%%%%%%%%%%%%%%%%
\chapter{Variable}


\begin{itemize}

\item \href{http://www1.unece.org/stat/platform/display/metis/Statistical+Variables+and+Characteristics}{Statistical
  Variables and Characteristics} (UNECE METIS wiki)

\item ``A variable is a characteristic of a statistical unit being observed that may assume more than one of a set of values to which a numerical measure or a category from a classification can be assigned.'' (\href{http://www.statcan.gc.ca/concepts/variable-eng.htm}{Statistics Canada})

\item ``A variable defines the concept of an observation (or measurement) for a given statistical unit type. The variable describes the concept of the observation that the data item results from. Thus, the variable is always associated with a contextual variable that describes the concept of the variable in the context of a particular statistical activity.'' (\href{http://www1.unece.org/stat/platform/download/attachments/14319930/Neuchatel+Model+V1.pdf?version=1}{Neuchâtel Terminology Model})

\item ``Definition: A characteristic of a unit being observed that may assume more than one of
a set of values.  Context: A variable in the mathematical sense, i.e. a quantity which may take any
one of specified set of values. It is convenient to apply the same word to
denote non-measurable characteristics, e.g., 'sex' is a variable in this sense
since any human individual may take one of two 'values', male or female. It
is useful, but far from being the general practice, to distinguish between a
variable as so defined and a random variable (The International Statistical
Institute, "The Oxford Dictionary of Statistical Terms", edited by Yadolah
Dodge, Oxford University Press, 2003).'' (\href{http://sdmx.org/wp-content/uploads/2009/01/04\_sdmx\_cog\_annex\_4\_mcv\_2009.pdf}{SDMX Statistical Guidelines Part IV Metadata Common Vocabulary})
\end{itemize}


%%%%%%%%%%%%%%%%%%%%%%%%%%%%%%%%%%%%%%%%%%%%%%%%%%%%%%%%%%%%%%%%
\chapter{Metadata}

\href{http://www.unece.org/stats/metis.html}{UNECE Statistical Metadata (METIS)}

\href{http://www1.unece.org/stat/platform/display/metis/The+Common+Metadata+Framework}{UNECE The Common Metadata Framework}

%%%%%%%%%%%%%%%%%%%%%%%%%%%%%%%%%%%%%%%%%%%%%%%%%%%%%%%%%%%%%%%%
\chapter{Codebook}

\begin{description}
  \item [Variable name]
  \item [Variable semantics] e.g. ``age in years''
  \item [Variable label] or description  (often an informal definition)
    \item [Variable ``level''] nominal, ordinal, interval, ratio
  \item [Variable types]
    \begin{description}
    \item [numeric]
      \begin{description}
      \item [integer]
      \item [real]
      \item [Unit of Measure] e.g. years
      \end{description}
    \item [enumerations] - code, label, meaning
    \item [dates]
    \item [durations]
    \item [etc.]
    \item [free text]
    \item [missing data]
    \end{description}
  \item [Semantic constraints] e.g. range constraints on age
\end{description}

%%%%%%%%%%%%%%%%%%%%%%%%%%%%%%%%%%%%%%%%%%%%%%%%%%%%%%%%%%%%%%%%
\chapter{Quality}

How is data quality defined?

